\documentclass{article}
\usepackage[utf8]{inputenc}
\usepackage{xcolor}

% Definisci il colore per le sezioni
\definecolor{subsectioncolor}{RGB}{0, 102, 204}

% Impostazioni per la pagina
\usepackage{geometry}
\geometry{a4paper, margin=1in}

% Pacchetto per hyperlink (facoltativo, per collegamenti interni)
\usepackage{hyperref}
\hypersetup{
    colorlinks=true,
    linkcolor=blue,
    urlcolor=blue,
    citecolor=blue
}

% Definizione dei colori
\definecolor{sectioncolor}{rgb}{0.2, 0.4, 0.6}
\definecolor{subsectioncolor}{rgb}{0.6, 0.2, 0.2}
\definecolor{textcolor}{rgb}{0.1, 0.1, 0.1}

\title{\textbf{Trascrizione intervista}\\ Utente Rappresentativo: Aurora}
\author{Carmen Giaccotto, Manoueil Michael Halim Riad Hanna \\ Typeset by Mattia Colombo}
\date{9 Ottobre 2024}

\begin{document}

\maketitle

\section{Introduzione}
Questa intervista è stata condotta con Aurora, studentessa universitaria di 19 anni, nell'ambito di una ricerca universitaria. 
L’obiettivo di questa intervista è quello di comprendere meglio l'esperienza e le percezioni rispetto alla fruizione di contenuti culturali da parte di ragazzi adolescenti, al fine di esplorare come la tecnologia possa essere integrata in modo efficace per rendere l’esperienza museale più accessibile e coinvolgente.

\subsection{\textcolor{subsectioncolor}{Intervista con Aurora}}

\begin{itemize}
    \item \textbf{Intervistatore:} Ciao! Sono Carmen e condurrò questa intervista insieme al mio collega Manoueil. Siamo un gruppo di ricerca del Politecnico di Milano nell’ambito della Human-Computer Interaction. Siamo laureandi in ingegneria informatica e la nostra missione è  quella di esplorare come poter coniugare tecnologia e cultura per renderle più accessibili e coinvolgenti, soprattutto per voi giovani. Potresti presentarti? Come ti chiami? Da dove vieni? Cosa studi?
    
    \item \textbf{Intervistato:} Mi chiamo Aurora, vengo da Cassano in provincia di Milano e ho appena iniziato il primo anno di Economia all’Università di Bergamo.
    
    \item \textbf{Intervistatore:} Cosa ti piace fare nel tuo tempo libero?
    
    \item \textbf{Intervistato:} Nel mio tempo libero mi piace disegnare, leggere e praticare sport.
    
    \item \textbf{Intervistatore:} Sei molto sportiva e interessata all’arte. Cosa ti ha spinto a partecipare a questo scambio di idee oggi?
    
    \item \textbf{Intervistato:} In primo luogo poter darvi un aiuto ed anche per confrontarmi su queste esperienze, per vedere se le ho vissute e avere un’idea su questo aspetto.
    
    \item \textbf{Intervistatore:} Ti è mai capitato di visitare un museo? Se sì, con che frequenza li visiti e quanto tempo dedichi?
    
    \item \textbf{Intervistato:} Solitamente visito i musei quando viaggio, soprattutto con la famiglia. Uno dei musei che mi è rimasto maggiormente impresso è stato il Museo del Louvre.
    
    \item \textbf{Intervistatore:} Bellissimo. Quindi le tue esperienze sono state fatte con familiari, non con persone della tua età?
    
    \item \textbf{Intervistato:} Esatto, sempre con la famiglia.
    
    \item \textbf{Intervistatore:} Ti è mai capitato di visitare un museo di tua iniziativa, senza essere in viaggio o con la famiglia?
    
    \item \textbf{Intervistato:} No, non mi è mai capitato.
    
    \item \textbf{Intervistatore:} E durante il liceo, hai mai partecipato a gite scolastiche che includevano visite a musei?
    
    \item \textbf{Intervistato:} No, che io ricordi, no.
    
    \item \textbf{Intervistatore:} Quando visiti musei con i tuoi familiari in vacanza, che tipo di musei preferisci? Arte, storia, altro?
    
    \item \textbf{Intervistato:} Per la maggior parte musei d’arte. A volte, spinta dalla curiosità, ho visitato mostre a Milano, come quella di Van Gogh o di arte contemporanea. Ma oltre a queste occasioni, visito musei soprattutto quando viaggio all’estero.
    
    \item \textbf{Intervistatore:} Quindi usi i musei anche per informarti sul luogo che stai visitando. Quando vai al museo, preferisci esplorarlo autonomamente o utilizzi guide fisiche o audioguide?
    
    \item \textbf{Intervistato:} Nella maggior parte dei casi utilizzo strumenti tecnologici come audioguide che mi spiegano ciò che vedo tramite auricolari. Non ho mai fatto una visita con una guida fisica o con un gruppo; mi muovo in modo autonomo ma guidata dalle spiegazioni.
    
    \item \textbf{Intervistatore:} C’è un aspetto dei musei che ti attira particolarmente? Ad esempio, attività interattive, laboratori o schermi multimediali?
    
    \item \textbf{Intervistato:} Nel caso del Louvre, non ci sono stati elementi particolarmente interattivi. Ma alla mostra di Van Gogh, alla fine del percorso c’era una sala immersiva con schermi che proiettavano contemporaneamente diverse parti della vita del pittore. Era una parte più interattiva rispetto al resto del percorso, che era più statico.
    
    \item \textbf{Intervistatore:} Grazie. Ora passo la parola a Manoueil, che ti farà alcune domande.
    
    \item \textbf{Intervistatore:} Ciao Aurora. Hai menzionato l’uso di audioguide e altri strumenti tecnologici durante le tue visite. Ci sono altri strumenti che hai trovato utili o interessanti?
    
    \item \textbf{Intervistato:} Recentemente, alla Sagrada Familia a Barcellona, oltre all’audioguida, c’era la possibilità di noleggiare un tablet. In determinati punti, inquadrando ad esempio una cupola, potevi vedere da vicino dettagli non visibili a occhio nudo grazie alla tecnologia. Non solo ascoltavi, ma vedevi anche elementi aggiuntivi.
    
    \item \textbf{Intervistatore:} Hai trovato questi strumenti facili da usare? Ti hanno aiutato a comprendere meglio le opere?
    
    \item \textbf{Intervistato:} Sì, assolutamente. Ti permettono di avere una visione più completa e di comprendere meglio ciò che osservi, rispetto al solo ascolto di una spiegazione.
    
    \item \textbf{Intervistatore:} Quindi consideri l’uso della tecnologia non come una distrazione, ma come un aiuto utile?
    
    \item \textbf{Intervistato:} Sì, sicuramente.
    
    \item \textbf{Intervistatore:} Ti è mai capitato di usare chatbot o cercare informazioni su Internet durante le visite, o hai sempre trovato tutto ben spiegato?
    
    \item \textbf{Intervistato:} No, non mi è mai capitato di usare chatbot. Ho sempre trovato tutto spiegato nei minimi dettagli.
    
    \item \textbf{Intervistatore:} Passiamo a domande più generali. Secondo te, cosa possono insegnare i musei ai ragazzi della tua età e perché?
    
    \item \textbf{Intervistato:} Permettono di vedere dal vivo ciò che si è studiato sui libri, offrendo un confronto con le proprie aspettative su un’opera. Inoltre, se si tratta di un artista poco conosciuto, arricchiscono il bagaglio culturale in modo più pratico rispetto allo studio teorico.
    
    \item \textbf{Intervistatore:} Consiglieresti a un tuo amico di visitare un museo? Come lo convinceresti?
    
    \item \textbf{Intervistato:} Dipende dal tipo di museo. Personalmente, non consiglierei il Museo del Louvre a chi non è appassionato, perché è molto vasto e può risultare dispersivo. Se invece si è interessati a un determinato artista, suggerirei di cogliere l’occasione quando c’è una mostra dedicata.
    
    \item \textbf{Intervistatore:} Come immagini il museo perfetto per ragazzi della tua età? Su quali aspetti dovrebbe puntare?
    
    \item \textbf{Intervistato:} Dovrebbe avere parti più interattive, come laboratori, e non limitarsi all’esposizione di quadri. Sarebbe utile permettere ai visitatori di interagire con le opere attraverso video interattivi o dispositivi che, inquadrando un quadro, mostrano immagini correlate.
    
    \item \textbf{Intervistatore:} Secondo te, un’applicazione dedicata all’esperienza museale sarebbe utile sia per organizzare la visita che per arricchirla?
    
    \item \textbf{Intervistato:} Sì, penso di sì.
    
    \item \textbf{Intervistatore:} Per rendere l’esperienza museale più divertente, hai idee su attività o funzionalità che potrebbero coinvolgere maggiormente i ragazzi?
    
    \item \textbf{Intervistato:} Potrebbe essere interessante avere stanze dedicate alla vita privata dell’artista, per conoscere il suo vissuto e staccare un po’ dalla semplice esposizione dei quadri.
    
    \item \textbf{Intervistatore:} Ci sono esperienze di altri luoghi di svago, come cinema o concerti, che vorresti vedere integrate nei musei?
    
    \item \textbf{Intervistato:} Per quanto riguarda la musica, penso che in determinate sale potrebbe aumentare l’impatto emotivo. Un accompagnamento musicale coerente con le opere esposte potrebbe coinvolgere maggiormente i giovani.
    
    \item \textbf{Intervistatore:} Hai qualche suggerimento per migliorare l’esperienza museale per i ragazzi della tua età?
    
    \item \textbf{Intervistato:} Trovo un po’ noiose le visite di gruppo. Sarebbe meglio focalizzarsi su aspetti che permettano a ciascuno di avere un’esperienza completa anche girando autonomamente nel museo. Inoltre, sarebbe utile avere dispositivi più sofisticati che non richiedano di scaricare un’app e usare i propri auricolari. In alcune sale potrebbero esserci apparecchiature che, a intervalli regolari, forniscano spiegazioni generali su ciò che si sta vedendo.
    
    \item \textbf{Intervistatore:} Grazie mille, Aurora. Con questo abbiamo terminato. Ora passo la parola a Carmen per la conclusione.
    
    \item \textbf{Intervistatore:} Grazie, Aurora. Sei stata molto esaustiva e le tue risposte sono preziose per la nostra ricerca. Ti ringraziamo e ti auguriamo una buona serata.
    
\end{itemize}

\end{document}