\documentclass{article}
\usepackage[utf8]{inputenc}
\usepackage{xcolor}

% Definisci il colore per le sezioni
\definecolor{subsectioncolor}{RGB}{0, 102, 204}

% Impostazioni per la pagina
\usepackage{geometry}
\geometry{a4paper, margin=1in}

% Pacchetto per hyperlink (facoltativo, per collegamenti interni)
\usepackage{hyperref}
\hypersetup{
    colorlinks=true,
    linkcolor=blue,
    urlcolor=blue,
    citecolor=blue
}

% Definizione dei colori
\definecolor{sectioncolor}{rgb}{0.2, 0.4, 0.6}
\definecolor{subsectioncolor}{rgb}{0.6, 0.2, 0.2}
\definecolor{textcolor}{rgb}{0.1, 0.1, 0.1}

\title{\textbf{Trascrizione intervista}\\ Utente Rappresentativo: Marco}
\author{Valentina Petrignano \\ Typeset by Mattia Colombo}
\date{10 Ottobre 2024}

\begin{document}

\maketitle

\section{Introduzione}
Marco è uno studente di 18 anni che frequenta il Liceo Scientifico. Nel suo tempo libero si dedica a molte attività come la cucina, lo sport, la musica e la lettura. L’obiettivo di questa intervista è comprendere le sue esperienze e percezioni rispetto alle visite museali e individuare soluzioni per rendere i musei più coinvolgenti per i giovani attraverso l’integrazione della tecnologia.

\subsection{\textcolor{subsectioncolor}{Intervista con Marco}}

\begin{itemize}

    \item \textbf{Intervistatore:} Ciao Marco, sono Valentina e sto lavorando a un progetto in cui dobbiamo trovare una soluzione tecnologica che possa incoraggiare i ragazzi a frequentare i musei più spesso, sentendosi più coinvolti. Oggi faremo una chiacchierata; ho una lista di domande. Non ci sono risposte giuste o sbagliate, quindi sentiti libero di esprimere la tua opinione. Ti puoi presentare? Chi sei, quanti anni hai, cosa studi e cosa ti piace fare nel tempo libero?

    \item \textbf{Intervistato:} Mi chiamo Marco, ho 18 anni. Studio al Liceo Scientifico Allende a Lambrate. Nel mio tempo libero mi piace cucinare, giocare a basket, fare sport in generale, suonare la chitarra o il pianoforte e leggere ogni tanto.

    \item \textbf{Intervistatore:} Hai mai visitato un museo?

    \item \textbf{Intervistato:} Sì.

    \item \textbf{Intervistatore:} Con che frequenza visiti i musei e quanto tempo dedichi generalmente a una visita?

    \item \textbf{Intervistato:} Circa quattro volte l’anno. Di solito, quando vado in un museo, ci sto dentro almeno due ore.

    \item \textbf{Intervistatore:} Di solito li visiti con la scuola, con la famiglia o con gli amici?

    \item \textbf{Intervistato:} Spesso vado con la mia famiglia, soprattutto quando siamo in vacanza. L’anno scorso, quando ero in Irlanda, ci sono stato diverse volte con amici. Dipende dal contesto: qui in Italia vado con la famiglia, in Irlanda andavo con amici.

    \item \textbf{Intervistatore:} Preferisci andare con qualcuno in particolare?

    \item \textbf{Intervistato:} In generale, preferisco andare con degli amici.

    \item \textbf{Intervistatore:} Quale modalità preferisci per seguire l’esperienza museale? In autonomia o con la guida di una persona esperta?

    \item \textbf{Intervistato:} Decisamente con una guida, preferibilmente una guida fisica. Anche un’audioguida va bene, ma se devo visitare un museo d’arte da solo senza guida, non sono molto contento perché non capisco il significato di ciò che sto guardando.

    \item \textbf{Intervistatore:} Consideri i musei come luoghi per imparare qualcosa o più per passare il tempo come hobby o svago?

    \item \textbf{Intervistato:} Li vedo più come luoghi dove imparare qualcosa.

    \item \textbf{Intervistatore:} C’è un museo che ti è particolarmente piaciuto? Cosa ti ha colpito?

    \item \textbf{Intervistato:} Sì, il Museo della Scienza e della Tecnica a Milano, perché è molto interattivo e pieno di attività per i visitatori. Apprezzo i musei in cui il visitatore può fare qualcosa, quelli interattivi.

    \item \textbf{Intervistatore:} C’è un museo che non ti è piaciuto? Cosa non ti ha convinto?

    \item \textbf{Intervistato:} Le visite più noiose che mi sono capitate sono state nei musei d’arte, come le pinacoteche, non tanto perché non mi piacessero i quadri, ma perché erano visite individuali senza guida. Guardavo solo delle immagini e non capivo perché ero lì.

    \item \textbf{Intervistatore:} C’è un aspetto dei musei che apprezzi particolarmente e che vorresti rimanesse invariato?

    \item \textbf{Intervistato:} Non mi dà fastidio se le persone parlano mentre guardano un’opera, magari discutendo a riguardo. Preferisco musei non troppo affollati, dove non c’è una coda infinita per vedere un’opera. Apprezzo anche quando ci sono panchine su cui sedersi davanti all’opera per poterla guardare meglio e più a lungo.

    \item \textbf{Intervistatore:} Hai mai utilizzato strumenti tecnologici durante una visita al museo, come audioguide, app o realtà aumentata? Li hai trovati accessibili e facili da usare?

    \item \textbf{Intervistato:} Sì, mi è capitato spesso di usare QR code da inquadrare accanto all’opera che rimandano ad audioguide. Sono facili da usare e li ho apprezzati. Al Museo della Scienza e della Tecnica ci sono anche monitor con giochi didattici.

    \item \textbf{Intervistatore:} Preferisci visitare mostre tradizionali o quelle che includono tecnologie moderne?

    \item \textbf{Intervistato:} Preferisco quelle che includono tecnologie moderne.

    \item \textbf{Intervistatore:} Ti piacerebbe vedere più tecnologia nei musei?

    \item \textbf{Intervistato:} Sì, assolutamente. Mi è capitato a Parigi di trovare una mostra in realtà aumentata, ma era separata dalla visita tradizionale. Preferisco quando la tecnologia è integrata con la mostra tradizionale e le opere fisiche.

    \item \textbf{Intervistatore:} C’è un tipo di tecnologia in particolare che vorresti fosse implementata maggiormente nei musei?

    \item \textbf{Intervistato:} Una volta ho visitato una mostra su Leonardo da Vinci dove c’erano pannelli con attori che raccontavano la sua biografia. Mi è piaciuto molto. Quando ci sono video che raccontano l’opera o la vita dell’autore, magari recitati da attori, è una cosa che apprezzo molto.

    \item \textbf{Intervistatore:} Hai mai avuto la possibilità di utilizzare chatbot o assistenti virtuali in un museo per fare domande o trovare informazioni?

    \item \textbf{Intervistato:} No, non mi è mai capitato.

    \item \textbf{Intervistatore:} Ti piacerebbe avere questa possibilità?

    \item \textbf{Intervistato:} Sì, probabilmente sì, anche se preferirei rivolgermi al personale del museo per chiedere informazioni.

    \item \textbf{Intervistatore:} Consiglieresti a un tuo amico di visitare un museo che ti ha colpito?

    \item \textbf{Intervistato:} Sì, lo consiglierei senz’altro se ne vale la pena.

    \item \textbf{Intervistatore:} Pensi che i musei più interattivi o moderni potrebbero essere più attraenti per i ragazzi della tua età?

    \item \textbf{Intervistato:} Sì, assolutamente. Un museo interattivo e tecnologico è sicuramente più attraente per un ragazzo o una ragazza della mia età.

    \item \textbf{Intervistatore:} Se dovessi descrivere il museo perfetto per ragazzi della tua età, come sarebbe?

    \item \textbf{Intervistato:} Avrebbe tutte le caratteristiche di cui ho parlato prima: visite guidate, attività da fare accanto alle opere, video che raccontano la storia dell’opera e la biografia dell’autore. Sarebbe ospitato in un edificio moderno con un bel design.

    \item \textbf{Intervistatore:} Ritieni che lo sviluppo di un’applicazione dedicata possa contribuire a organizzare meglio l’esperienza museale?

    \item \textbf{Intervistato:} Sì, penso che possa aiutare. Sarebbe utile un’applicazione che raggruppa diversi musei, non un’app diversa per ogni museo. Questo renderebbe più pratico avere tutte le informazioni in un unico posto.

    \item \textbf{Intervistatore:} Pensi che funzionalità di gamification in un’app potrebbero rendere più divertente e facile l’esperienza museale per te e i tuoi amici?

    \item \textbf{Intervistato:} Sì, sarebbe una buona idea. Ad esempio, una sfida o un quiz da completare insieme agli amici durante la visita renderebbe l’esperienza più divertente. Mi è capitato a Roma, durante una visita su Caravaggio, di partecipare a un’attività simile e l’ho trovata molto divertente.

    \item \textbf{Intervistatore:} C’è qualcos’altro che vorresti aggiungere per migliorare l’esperienza di visita in un museo, sempre rivolto a ragazzi della tua età?

    \item \textbf{Intervistato:} Oltre alle sfide e ai quiz, mi piacerebbe vedere più laboratori alla fine della visita, che possono andare da piccoli lavoretti a qualcosa di più particolare da portare a casa. Penso che sarebbe interessante sia per gli adolescenti che per i bambini più piccoli.

    \item \textbf{Intervistatore:} Grazie mille, Marco. Abbiamo finito. Ti ringrazio per il tuo tempo.

    \item \textbf{Intervistato:} Grazie a te.

\end{itemize}

\end{document}