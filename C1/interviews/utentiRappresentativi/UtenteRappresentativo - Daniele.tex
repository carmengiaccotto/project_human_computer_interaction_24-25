\documentclass{article}
\usepackage[utf8]{inputenc}
\usepackage{xcolor}

% Definisci il colore per le sezioni
\definecolor{subsectioncolor}{RGB}{0, 102, 204}

% Impostazioni per la pagina
\usepackage{geometry}
\geometry{a4paper, margin=1in}

% Pacchetto per hyperlink (facoltativo, per collegamenti interni)
\usepackage{hyperref}
\hypersetup{
    colorlinks=true,
    linkcolor=blue,
    urlcolor=blue,
    citecolor=blue
}

% Definizione dei colori
\definecolor{sectioncolor}{rgb}{0.2, 0.4, 0.6}
\definecolor{subsectioncolor}{rgb}{0.6, 0.2, 0.2}
\definecolor{textcolor}{rgb}{0.1, 0.1, 0.1}

\title{\textbf{Trascrizione intervista}\\ Utente Rappresentativo: Daniele}
\author{Valentina Petrignano, Federico Previtali \\ Typeset by Mattia Colombo}
\date{9 Ottobre 2024}

\begin{document}

\maketitle

\section{Introduzione}
Questa intervista è stata condotta con Daniele, studente universitario di 19 anni e insegnante di teatro per bambini, nell'ambito di una ricerca universitaria. 
L’obiettivo di questa intervista è quello di comprendere meglio l'esperienza e le percezioni rispetto alla fruizione di contenuti culturali da parte di ragazzi adolescenti, al fine di esplorare come la tecnologia possa essere integrata in modo efficace per rendere l’esperienza museale più accessibile e coinvolgente.

\subsection{\textcolor{subsectioncolor}{Intervista con Daniele}}

\begin{itemize}
    \item \textbf{Intervistatore:} Ciao! Sono Valentina e condurrò questa intervista insieme al mia collega Federico. Siamo studenti del Politecnico di Milano e facciamo parte del gruppo “Designer for Culture”. Stiamo facendo una ricerca per esplorare e sviluppare soluzioni innovative che uniscano tecnologia e cultura, rendendo l’arte e il patrimonio culturale più accessibili e coinvolgenti, soprattutto per adolescenti e giovani.

    Oggi faremo una chiacchierata informale, quindi nessuna ansia. Non ci sono risposte giuste o sbagliate; l’importante è che tu ci dica ciò che pensi realmente.

    Per iniziare, potresti presentarti? Chi sei? Cosa studi? Cosa fai?

    \item \textbf{Intervistato:} Mi chiamo Daniele, sono del 2005, a luglio mi sono diplomato al liceo classico Tito Livio di Milano ed ho appena iniziato  a frequentare la facoltà di psicologia in Bicocca. Ho anche iniziato a lavorare questa settimana in un corso di teatro, dove insegno a bambini delle elementari.

    \item \textbf{Intervistatore:} Hai mai visitato un museo?

    \item \textbf{Intervistato:} Sì, certo.

    \item \textbf{Intervistatore:} Con quale frequenza visiti i musei e quanto tempo dedichi generalmente alla visita?

    \item \textbf{Intervistato:} Dipende dai periodi. Quando ho più tempo libero, ci vado più spesso, anche una volta a settimana con i miei amici. Altre volte, quando sono impegnato con lo studio, non ci vado. Direi comunque abbastanza spesso.

    \item \textbf{Intervistatore:} Di solito li visiti con la scuola o preferisci andarci per conto tuo, per svago o interesse?

    \item \textbf{Intervistato:} Li ho visitati con la scuola e siamo stati fortunati perché ci siamo andati spesso. Tuttavia, preferisco andare per conto mio con i miei amici, slegato dal contesto scolastico. In ogni caso mi diverto anche durante le gite scolastiche.

    \item \textbf{Intervistatore:} Come sono state le tue esperienze con la scuola?

    \item \textbf{Intervistato:} Non negative, ma avevamo un’insegnante di storia dell’arte che faceva le spiegazioni lei stessa e non era proprio bravissima. Faceva errori storici che noi studenti notavamo e spesso non sapeva rispondere alle nostre domande. Non era molto stimolante. Con guide professionali, invece, le visite sono state più interessanti e stimolanti.

    \item \textbf{Intervistatore:} Preferisci andare in autonomia o con una guida esperta?

    \item \textbf{Intervistato:} Preferisco con una guida. Sicuramente meglio rispetto all’autonomia o all’uso di audioguide, che non riesco a seguire bene. Con una guida posso interagire e fare domande.

    \item \textbf{Intervistatore:} Consideri i musei luoghi puramente didattici dove si imparano cose nuove o posti dove vai per hobby e svago?

    \item \textbf{Intervistato:} Direi più la seconda. Spesso le mostre che visito riguardano artisti o opere che ho già studiato e conosciuto. Mi piace guardare un’opera d’arte piuttosto che analizzarla nei minimi dettagli, perché credo che l’osservazione diretta dia qualcosa in più.

    \item \textbf{Intervistatore:} C’è stato un museo o una mostra che ti è piaciuta particolarmente o che non ti ha convinto?

    \item \textbf{Intervistato:} Mi è piaciuta molto la mostra di Banksy alla Stazione Centrale, perché amo la street art. Un’altra mostra che ho apprezzato è stata “Corpus Domini” a Palazzo Reale, con installazioni di arte contemporanea molto belle. Non ho avuto esperienze negative in generale.

    \item \textbf{Intervistatore:} Ti senti coinvolto quando visiti un museo o pensi che siano luoghi passivi e distanti dai tuoi interessi?

    \item \textbf{Intervistato:} Mi sento coinvolto. Non li percepisco come distanti dai miei interessi. Non sento il bisogno di ulteriori attività per sentirmi coinvolto; mi piace l’idea di essere da solo davanti all’opera di un artista.

    \item \textbf{Intervistatore:} C’è un aspetto dei musei che apprezzi particolarmente e che vorresti rimanesse invariato? O qualcosa che cambieresti?

    \item \textbf{Intervistato:} Apprezzo molto l’iniziativa dei musei gratuiti ogni primo mercoledì del mese; è fantastica. Mi piace anche l’analisi che i direttori fanno nel decidere il percorso espositivo basato sulla loro interpretazione dell’artista. A volte concordo con le loro scelte, altre volte meno, ma è interessante vedere diverse interpretazioni.

    \item \textbf{Intervistatore:} Hai detto che preferisci visitare i musei con una guida. Ti è mai capitato di utilizzare strumenti tecnologici durante una visita? Se sì, quali?

    \item \textbf{Intervistato:} Ho provato le audioguide, ma non mi piacciono. Faccio fatica a seguire il ritmo e non posso interagire. Con una guida posso fare domande; con l’audioguida mi sento lasciato a me stesso, cosa che va bene se conosco già l’artista, altrimenti no.

    \item \textbf{Intervistatore:} Preferisci mostre tradizionali o che includono tecnologia moderna?

    \item \textbf{Intervistato:} Non saprei dire. Ho visto una mostra di Van Gogh in realtà aumentata, ma non mi ha entusiasmato. Forse preferisco il museo tradizionale.

    \item \textbf{Intervistatore:} Pensi che sarebbe positivo utilizzare più tecnologia nei musei?

    \item \textbf{Intervistato:} Sì, ma dipende dalla quantità. Se visito una mostra su un artista del ‘700, non mi aspetto molta tecnologia; potrebbe sembrare anacronistico. La tecnologia deve essere usata in contesti e modi appropriati. Non vorrei vedere i quadri di Van Gogh scorrere sotto i miei piedi su uno schermo OLED; mi sembrerebbe inutile. Tuttavia, un’integrazione ponderata potrebbe essere utile, soprattutto per coinvolgere i giovani, purché non sia eccessiva o fuori luogo.

    \item \textbf{Intervistatore:} Hai mai utilizzato chatbot o assistenti virtuali durante una visita al museo?

    \item \textbf{Intervistato:} No, mai.

    \item \textbf{Intervistatore:} Pensi che potrebbe essere utile?

    \item \textbf{Intervistato:} Potrebbe essere utile per ottenere informazioni, ma come studente di psicologia preferisco interagire con una persona reale. Un chatbot mi sembrerebbe freddo e distante dall’arte, che per me è qualcosa di emozionale. Sarebbe utile, ma non l’ideale.

    \item \textbf{Intervistatore:} Consiglieresti a un amico di visitare un museo? Come lo convinceresti?

    \item \textbf{Intervistato:} Penso che sia una scelta personale. Quando ho visto “Corpus Domini”, ne ho parlato ai miei amici dicendo che mi era piaciuta molto. Alcuni hanno deciso di andarci perché interessati, ma non vado in giro a dire “devi assolutamente vederla”.

    \item \textbf{Intervistatore:} Come sarebbe il museo perfetto per un ragazzo della tua età?

    \item \textbf{Intervistato:} Per me dovrebbe essere in un edificio storico, con un’architettura artistica che ti immerge nell’ambiente fin dall’inizio. Preferisco un museo tradizionale, con diverse sale, installazioni e opere d’arte, e una guida.

    \item \textbf{Intervistatore:} Pensi che lo sviluppo di un’applicazione dedicata possa contribuire alla visita e all’organizzazione dell’esperienza museale?

    \item \textbf{Intervistato:} Sì, soprattutto per l’organizzazione. Spesso è complicato reperire informazioni o organizzare la visita. Un’app che integra tutto sarebbe più comoda.

    \item \textbf{Intervistatore:} Pensi che funzionalità di gamification in un’app potrebbero rendere l’esperienza museale più divertente per i giovani?

    \item \textbf{Intervistato:} Potrebbe essere divertente, ad esempio, fare un quiz tipo Kahoot dopo la visita, magari durante una gita scolastica, per verificare quanto si è appreso. Altri giochi non mi vengono in mente, ma potrebbe essere interessante un questionario sulle opere che hanno colpito di più.

    \item \textbf{Intervistatore:} Ci sono esperienze in altri luoghi di svago, come cinema o concerti, che vorresti trovare anche nei musei?

    \item \textbf{Intervistato:} Al momento non mi viene in mente nulla.

    \item \textbf{Intervistatore:} Hai qualche altro suggerimento per migliorare l’esperienza museale per i giovani?

    \item \textbf{Intervistato:} Nel questionario ho proposto un sistema di recensioni in un’applicazione, diviso in due sezioni: una parte con critiche di esperti e una con opinioni del pubblico. Le recensioni più positive potrebbero essere evidenziate, creando una sorta di classifica. Questo aiuterebbe le persone a scoprire mostre diverse e a scegliere in base alle proprie preferenze.

    \item \textbf{Intervistatore:} Preferiresti un’applicazione o un sito web per questo scopo?

    \item \textbf{Intervistato:} Preferirei un’applicazione.

    \item \textbf{Intervistatore:} L’intervista è conclusa. Grazie mille, sei stato molto d’aiuto e ci hai fornito risposte preziose.
\end{itemize}

\end{document}