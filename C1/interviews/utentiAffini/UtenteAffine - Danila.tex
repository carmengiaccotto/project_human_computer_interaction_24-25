\documentclass{article}
\usepackage[utf8]{inputenc}
\usepackage{xcolor}

% Definisci il colore per le sezioni
\definecolor{subsectioncolor}{RGB}{0, 102, 204}

% Impostazioni per la pagina
\usepackage{geometry}
\geometry{a4paper, margin=1in}

% Pacchetto per hyperlink (facoltativo, per collegamenti interni)
\usepackage{hyperref}
\hypersetup{
    colorlinks=true,
    linkcolor=blue,
    urlcolor=blue,
    citecolor=blue
}

% Definizione dei colori
\definecolor{sectioncolor}{rgb}{0.2, 0.4, 0.6}
\definecolor{subsectioncolor}{rgb}{0.6, 0.2, 0.2}
\definecolor{textcolor}{rgb}{0.1, 0.1, 0.1}

\title{\textbf{Trascrizione intervista}\\ Utente Affine: Danila}
\author{Carmen Giaccotto \\ Typeset by Mattia Colombo}
\date{9 Ottobre 2024}

\begin{document}

\maketitle

\section{Introduzione}
Questa intervista è stata condotta con Danila, genitore di un adolescente di 18 anni.
L'obiettivo di questa intervista è quello di comprendere meglio l'esperienza e le percezioni di un genitore rispetto alla fruizione di contenuti culturali da parte del proprio figlio adolescente, al fine di esplorare come la tecnologia possa essere integrata in modo efficace per rendere l’esperienza museale più accessibile e coinvolgente per i ragazzi.

\subsection{\textcolor{subsectioncolor}{Intervista con Danila}}

\begin{itemize}
    \item \textbf{Intervistatore:} Buonasera, mi chiamo Carmen e sono una studentessa di ingegneria informatica. Sto conducendo una ricerca nell’ambito della Human-Computer Interaction con il Politecnico di Milano. La nostra missione è esplorare come la tecnologia e la cultura possano integrarsi per rendere il patrimonio culturale più accessibile e coinvolgente, soprattutto per gli adolescenti.\\
    
    Le vorrei chiedere innanzitutto se può presentarsi. Ci dice quanti figli ha, le loro età, qual è la sua occupazione, i suoi interessi ed hobby?
    
    \item \textbf{Intervistato:} Certo. Ho due figli: uno di 15 anni che si chiama Lorenzo e uno di 18 che si chiama Francesco. Entrambi frequentano il liceo classico. Io lavoro nella segreteria di una scuola come amministrativa. Per quanto riguarda i miei interessi, mi piace molto camminare, fare giardinaggio e volontariato in parrocchia.
    
    \item \textbf{Intervistatore:} I suoi figli hanno mai visitato un museo? Nel caso, potrebbe parlarci della frequenza con cui li visitano e se ha notato un particolare interesse da parte loro per determinati tipi di musei?
    
    \item \textbf{Intervistato:} Sì, spesso vanno con le gite d’istruzione scolastiche. Qualche volta, quando siamo in viaggio, ci soffermiamo a visitare dei musei. A loro piacciono molto i Musei di Scienza e Tecnologia, ma anche le opere d’arte di autori importanti come Raffaello, Michelangelo o pittori molto rinomati del Rinascimento.
    
    \item \textbf{Intervistatore:} Come mai pensa che questi musei siano interessanti per loro?
    
    \item \textbf{Intervistato:} Perché hanno studiato a scuola i vari pittori e hanno approfondito questi argomenti. Inoltre, sono interessati alle opere realizzate da scienziati e tecnici italiani e stranieri che si sono distinti, come Leonardo da Vinci con tutti i suoi progetti.
    
    \item \textbf{Intervistatore:} Sono quindi molto interessati principalmente perché possono vedere dal vivo ciò che hanno studiato a scuola.
    
    \item \textbf{Intervistato:} Sì, esatto. Inoltre, li incuriosisce provare e sperimentare il funzionamento degli attrezzi che uno scienziato aveva inventato. È bello trovare nei musei la possibilità di interagire con le opere fatte da questi grandi personaggi.
    
    \item \textbf{Intervistatore:} Pensa che queste esperienze possano contribuire alla loro crescita culturale e personale?
    
    \item \textbf{Intervistato:} Sì, certamente. Non solo guardare l’opera, ma anche analizzarla più dettagliatamente. Sarebbe bello avere delle descrizioni più approfondite quando si va al museo. Non sempre è facile trovare una guida disponibile che ti aiuti. Magari avere strumenti tecnologici o di intelligenza artificiale che siano di facile utilizzo, per approfondire la conoscenza di quell’opera con dettagli più particolari. Così possono approfondire ciò che hanno già studiato sui libri, guardandolo da vicino e avendo qualcuno che fa notare aspetti non riportati nei testi.
    
    \item \textbf{Intervistatore:} In maniera più interattiva quindi, e in modo meno serioso, senza l’austerità che c’è tra i banchi di scuola.
    
    \item \textbf{Intervistato:} Esatto, sì.
    
    \item \textbf{Intervistatore:} Quando i suoi figli visitano i musei, ci sono zone che tendono ad annoiarli di più o esperienze di visita che non hanno apprezzato?
    
    \item \textbf{Intervistato:} Sì, è capitato. Mi hanno raccontato che entrando nei musei, dopo un po’ si sono annoiati perché vedevano una mole enorme di oggetti e opere, ma che rimanevano insignificanti per loro perché non conoscevano magari l’autore o le tecniche. Non riuscivano a cogliere la bellezza o la genialità di quell’opera, perché magari non hanno fatto studi artistici. Avere qualcuno che glieli spiega, che fa notare i particolari, renderebbe la visita più entusiasmante. È capitato che siano andati al museo senza avere nessuno che li supportasse e sono usciti delusi.
    
    \item \textbf{Intervistatore:} Quindi pensa che potrebbero essere più interessati ad attività collaborative, in cui c’è più interazione, magari con strumenti di intelligenza artificiale che possano stimolarli di più e dare informazioni che normalmente non si conoscono?
    
    \item \textbf{Intervistato:} Sì, sicuramente. Anche perché, lo sappiamo, molto spesso i giovani sono pigri; magari non si vogliono soffermare a leggere lunghe descrizioni e si annoiano. Invece, utilizzando strumenti più interattivi, che li mantengano attenti e concentrati, riescono ad apprezzare meglio l’opera e a seguire con maggiore interesse.
    
    \item \textbf{Intervistatore:} Cosa potrebbe renderli ancora più interessati? Spesso si riscontra che nei musei vengono effettuate molte attività per attirare i giovani, ma non hanno grande successo. Magari riscuotono più successo tra le famiglie con bambini più piccoli o tra adulti oltre i 25 anni. Cosa potrebbe attirarli di più?
    
    \item \textbf{Intervistato:} Sicuramente anche degli incentivi da parte dello Stato, come bonus o sconti per entrare nei musei. I giovani, lo sappiamo, sono mantenuti dai genitori e non sempre possono permettersi di entrare nei musei con i costi attuali. Avere una scontistica maggiore o un bonus che permetta loro di visitare i musei quasi gratuitamente potrebbe invogliarli. Poi, magari, avere una sorta di pubblicità attraverso siti o app dedicate alle visite nei musei destinate ai giovani, dove si sollecita l’idea di andare perché in quel periodo c’è una mostra specifica o si fanno attività interattive non presenti tutto l’anno. In questo modo potrebbero essere più attratti e frequentare più volentieri i musei.
    
    \item \textbf{Intervistatore:} Quali sono le attività che normalmente vede i suoi figli più impegnati durante la giornata, oltre alla scuola? Ci sono attività che svolgono con molta passione, in cui li vede molto entusiasti?
    
    \item \textbf{Intervistato:} Sì, le attività sportive. A uno piace il calcio e all’altro il tennis. Quando c’è da andare a fare una partita, non se lo fanno dire due volte. Questo perché hanno modo di frequentare i coetanei e scaricano un po’ di adrenalina facendo sport. Poi tornano ricaricati.
    
    \item \textbf{Intervistatore:} Quindi, per il museo, potrebbe suggerire di organizzare pacchetti dedicati ai giovani, dove magari si incontrano tramite un’app con lo stesso interesse, creano un gruppo e visitano insieme, facilitando anche la gestione da parte del museo e riducendo i costi?
    
    \item \textbf{Intervistato:} Esatto, attività in cui sono in compagnia.
    
    \item \textbf{Intervistatore:} Le è mai capitato di usare sistemi digitali o tecnologici per acquistare i biglietti o accedere ad altri tipi di attività che un museo organizza? Ad esempio, per noleggiare audioguide?
    
    \item \textbf{Intervistato:} Per acquistare i biglietti online sì, l’ho utilizzato e lo trovo molto comodo. Francamente, andare in biglietteria a fare la fila stanca tutti, anche perché si va magari in posti che non sono il proprio luogo di residenza e si cerca di utilizzare al meglio il tempo. Quindi, evitare la fila al botteghino permette di dedicare più tempo alla visita del museo e della città in generale. Lo trovo molto utile. Non ho avuto la possibilità di utilizzare altre app per audioguide o prenotare in maniera digitale, ma lo riterrei sicuramente utile.
    
    \item \textbf{Intervistatore:} Nella città in cui vive, come vengono promosse le attività culturali e le esperienze nei musei? Vengono promosse esclusivamente tramite la scuola o ci sono altri modi in cui i giovani vengono a conoscenza di queste attività?
    
    \item \textbf{Intervistato:} Vivo in un piccolo paese; ci sono dei piccoli musei che conosciamo perché siamo in una realtà ristretta. Vedo che non sono molto pubblicizzati. Si promuovono di più durante le feste locali: allora si aprono e si rendono fruibili a tutti. Ma durante l’anno spesso li troviamo chiusi durante la settimana e aprono solo la domenica. Dovrebbero essere migliorati, avere più giornate di apertura al pubblico e trovare sistemi più tecnologici per far conoscere questi musei, che pur essendo piccoli, contengono oggetti preziosi e interessanti anche per chi non è del luogo.
    
    \item \textbf{Intervistatore:} Pensa che avere un ricco bagaglio culturale possa essere importante anche per diventare un cittadino più attivo, più consapevole e quindi più partecipativo all’interno del proprio comune o città?
    
    \item \textbf{Intervistato:} Sì, certo. Perché conoscere la propria storia e le proprie radici, le belle cose realizzate dai nostri artigiani, scultori, pittori e così via, certamente ci arricchisce. Veniamo a conoscenza di chi sono stati i nostri avi, qual è la nostra storia, quali sono le nostre radici. Senza di esse, un popolo rimane anonimo, non ha particolarità, non si distingue. Invece, conoscere le proprie radici e la bellezza delle opere realizzate può anche unire la comunità, farla sentire partecipe.
    
    \item \textbf{Intervistatore:} Vuole aggiungere qualcosa? Ha qualche consiglio in generale da darci su questo tema, sull’ambito dei musei, sugli interessi dei giovani verso la cultura e su come invogliarli maggiormente, oltre a quanto abbiamo già discusso?
    
    \item \textbf{Intervistato:} Intanto mi complimento con voi giovani che avete queste idee brillanti, perché la cultura è stata per troppo tempo trascurata. Parlo per la mia esperienza locale: musei davvero ricchi di opere di inestimabile valore, ma che non sono fruibili, non sono messi a conoscenza di tutti e non vengono pubblicizzati. Quindi, sicuramente fare molta più pubblicità attraverso i sistemi tecnologici, per raggiungere soprattutto i giovani. Utilizzare app, come abbiamo già detto, ma lo sottolineo. Anche attraverso i social, creare gruppi specifici di musei dove si invitano i giovani di una certa fascia d’età a iscriversi, per poi ritrovarsi tra loro con interessi comuni e fare squadra, invogliandosi a vicenda. Questo potrebbe essere un sistema molto utile.\\
    Inoltre, utilizzare maggiormente prodotti dell’intelligenza artificiale che riducono magari il numero di dipendenti nei musei, con una conseguente riduzione dei costi di gestione, offrendo al contempo un servizio in più. L’utilizzo di tecnologie avanzate ha un costo iniziale, ma poi rimane nel tempo e si ammortizza negli anni, garantendo un servizio migliore e più arricchente per tutti.
    
    \item \textbf{Intervistatore:} La ringrazio. Le sue parole sono molto preziose per la nostra ricerca. Le auguro una buona serata.
    
    \item \textbf{Intervistato:} Io vi auguro un buon lavoro e tanti auguri. Buona serata.
    
\end{itemize}

\end{document}