\documentclass{article}
\usepackage[utf8]{inputenc}
\usepackage{xcolor}

% Definisci il colore per le sezioni
\definecolor{subsectioncolor}{RGB}{0, 102, 204}

% Impostazioni per la pagina
\usepackage{geometry}
\geometry{a4paper, margin=1in}

% Pacchetto per hyperlink (facoltativo, per collegamenti interni)
\usepackage{hyperref}
\hypersetup{
    colorlinks=true,
    linkcolor=blue,
    urlcolor=blue,
    citecolor=blue
}

% Definizione dei colori
\definecolor{sectioncolor}{rgb}{0.2, 0.4, 0.6}
\definecolor{subsectioncolor}{rgb}{0.6, 0.2, 0.2}
\definecolor{textcolor}{rgb}{0.1, 0.1, 0.1}

\title{Transcript \\ Intervista con Carlo}
\author{Carmen Giaccotto, Valentina Petrignano}
\date{07 Ottobre 2024}

\begin{document}

\maketitle

\section{Introduzione}
Questa intervista è stata condotta con Carlo, Presidente dell’Archeoclub di Siracusa e membro del consiglio direttivo dell’Associazione Nazionale Guide Turistiche, nell'ambito di una ricerca universitaria. L'obiettivo dell'intervista è esplorare le dinamiche e le sfide nel coinvolgimento degli adolescenti nelle esperienze museali, con particolare riferimento all'uso della tecnologia e alle barriere che possono incontrare.

\subsection{\textcolor{subsectioncolor}{Intervista con Carlo}}

\begin{itemize}
    \item \textbf{Intervistatore:} Buongiorno, sono Carmen e condurrò questa intervista insieme alla mia collega Valentina. Siamo un gruppo di ricerca che sta lavorando su un progetto universitario. Siamo laureandi in Ingegneria Informatica e abbiamo deciso di approfondire come tecnologia e cultura possano essere integrate per rendere l'arte e il patrimonio culturale più accessibili e coinvolgenti, soprattutto per la fascia d'età tra i 16 e i 19 anni, quindi gli adolescenti.
    
    Vorremmo chiederle se può presentarsi e spiegarci qual è il suo ruolo all’interno del museo.
    
    \item \textbf{Intervistato:} Mi chiamo Carlo e sono una guida turistica da 45 anni. Sono vicepresidente dell’associazione Siracusa e dell’associazione regionale, e faccio parte del direttivo nazionale. Ho lavorato al Museo Paolo Orsi per 25 anni, inizialmente con una cooperativa, occupandomi del restauro. Dopo essere andato in pensione, ora faccio la guida turistica a tempo pieno.
    
    \item \textbf{Intervistatore:} Quindi lavora in questo ambito da tantissimo tempo. Ha visto nascere il museo e ha contribuito alla sua inaugurazione.
    
    \item \textbf{Intervistato:} Esatto. Lavoravo al museo già prima della sua apertura nel 1988. Dal 1985 ho partecipato all’organizzazione dell’apertura del Museo Paolo Orsi. Ricordo bene gli ultimi dieci giorni prima dell’inaugurazione: abbiamo lavorato come se fossero tre mesi concentrati in dieci giorni, perché le cose si fanno sempre all’ultimo momento. Il museo ha aperto il 16 gennaio 1988.
    
    \item \textbf{Intervistatore:} Cosa l’ha spinta a scegliere questa professione?
    
    \item \textbf{Intervistato:} Ho iniziato a lavorare nel museo a 27-28 anni. In quel periodo, le cooperative erano molto diffuse, e ci siamo costituiti in cooperativa trovando opportunità di lavoro attraverso una legge regionale siciliana che favoriva le cooperative. È stata una necessità, ma anche per amore verso la mia città e il suo patrimonio. Ho studiato per diventare guida turistica, una professione che mi è piaciuta molto. Sono laureato in lingue e ho quasi completato la laurea in lettere classiche; mi mancava solo la tesi, che purtroppo non ho potuto terminare.
    
    \item \textbf{Intervistatore:} Ha una carriera davvero ricca. Grazie alla sua professione ha potuto conoscere molte persone importanti?
    
    \item \textbf{Intervistato:} Sì, nel corso di 45 anni ho avuto l’opportunità di fare da guida a personaggi come la regina di Danimarca Margherita II, Baldovino e Paola del Belgio, Dario Fo, Pietro Mennea e molti altri. Spesso si incontrano persone molto importanti.
    
    \item \textbf{Intervistatore:} Non immaginavo che questo lavoro potesse offrire tali opportunità.
    
    \item \textbf{Intervistato:} Certo, hai la possibilità di fare da guida a tutti, dalla A alla Z, dal Papa al cliente comune.
    
    \item \textbf{Intervistatore:} Voi guide siete le persone più qualificate. L’abilità della guida sta proprio nel capire chi si ha di fronte e nel comunicare adeguatamente in base al livello culturale dell’interlocutore.
    
    \item \textbf{Intervistato:} Esattamente. La cosa più bella è anche parlare con i bambini: si lavora con scuole elementari, dalla prima alla terza classe. Una cosa è parlare a una scuola elementare, un’altra è rivolgersi a un ministro, a uno studioso o a un professore. Bisogna adattare il tipo di comunicazione al pubblico che si ha davanti.
    
    \item \textbf{Intervistatore:} Proprio su questo argomento, vorrei chiederle: come descriverebbe l’interesse dei visitatori adolescenti per le esposizioni? Incontra delle sfide particolari con questo tipo di gruppi quando fa da guida?
    
    \item \textbf{Intervistato:} L’interesse dei ragazzi dipende molto dalla professionalità e dalla qualità degli insegnanti che li accompagnano. Ci sono ragazzi che vengono già preparati, sanno cosa devono vedere e fanno domande pertinenti. Purtroppo, però, la maggioranza non è così preparata e viene solo per fare una vacanza a Siracusa. C’è una differenza enorme tra le scuole. Ad esempio, durante le rappresentazioni classiche nel mese di maggio, abbiamo molti licei classici che vengono a vedere le tragedie. Gli studenti del liceo classico sanno di cosa si parla, sanno cosa devono vedere e mostrano molto interesse, a differenza di studenti di istituti tecnici o professionali.
    
    \item \textbf{Intervistatore:} Il museo organizza attività fatte appositamente per attirare questo tipo di ragazzi?
    
    \item \textbf{Intervistato:} Organizza attività didattiche attraverso il proprio personale, ma non coinvolge le guide turistiche. Ogni museo è gestito da società private, soprattutto per quanto riguarda la biglietteria, il bookshop, i gadget, i bar e anche le visite didattiche. C’è una legge che permette ai privati, anche se non sono guide turistiche ma archeologi o storici dell’arte, di fare visite didattiche per le scuole. Ciò non toglie che una guida turistica possa essere chiamata direttamente dalla scuola per effettuare la visita. Ne facciamo molte in questo senso.
    
    \item \textbf{Intervistatore:} Queste attività organizzate riscontrano interesse?
    
    \item \textbf{Intervistato:} Non lo so con precisione, perché non le svolgo io. Le gestiscono gli impiegati delle società private che gestiscono i musei, come ad esempio Civita, che organizza le attività didattiche sia al Museo di Siracusa sia al Parco Archeologico.
    
    \item \textbf{Intervistatore:} Secondo lei, l’esperienza museale attuale è abbastanza lontana dalle preferenze dei giovani? Si può fare qualcosa per attirare di più i ragazzi, anche nel modo in cui è organizzato il museo, non solo a livello di attività?
    
    \item \textbf{Intervistato:} Tutto è migliorabile. Rispetto al passato, dove non c’era nulla, il nostro Museo Paolo Orsi è stato all’avanguardia fin dalla sua inaugurazione nel 1988. Nessun altro museo in Italia aveva i supporti didattici che aveva il Paolo Orsi: molte mappe, didascalie e pannelli esplicativi. Ad esempio, per quanto riguarda la colonizzazione greca, avevamo mappe che facevano capire quale colonia siciliana era stata fondata da quale madrepatria, utilizzando colori corrispondenti. Questi supporti aiutano a comprendere molte cose. Ora ce ne sono ancora di più, incluse immagini e rappresentazioni delle necropoli che mostrano come venivano tumulati i morti. Il Museo Paolo Orsi è sempre stato attrezzatissimo e all’avanguardia, anche perché è stato costruito appositamente per essere un museo, a differenza di altri che sono stati ricavati da vecchi palazzi. Questo ha permesso di avere vetrine e spazi progettati per i vari reperti.
    
    \item \textbf{Intervistatore:} Secondo lei, ci sono delle barriere che possono limitare un adolescente a visitare un museo? Ad esempio, il prezzo del biglietto o altre difficoltà in generale?
    
    \item \textbf{Intervistato:} Il museo offre alcune agevolazioni. Ogni prima domenica del mese l’ingresso è gratuito per tutti. Inoltre, c’è la possibilità di fare un biglietto cumulativo tra il Museo Archeologico e il Parco Archeologico. Il museo costa 10 euro, il parco 16,50 euro (attualmente, perché c’è una mostra; di solito era 13 euro). Separatamente costerebbero 23 euro, ma con il biglietto cumulativo si risparmia circa 5 euro. Quello che manca è un biglietto per famiglie. Una famiglia con due o tre figli potrebbe spendere 40-50 euro, che è una cifra significativa. Si potrebbe pensare di introdurre biglietti familiari, come fanno in altri paesi come la Spagna, dove ad esempio al Museo del Prado ci sono orari serali in cui l’ingresso è gratuito.
    
    \item \textbf{Intervistatore:} Quindi si potrebbe organizzare una sorta di “notte al museo” con prezzi ridotti.
    
    \item \textbf{Intervistato:} Esatto.
    
    \item \textbf{Intervistatore:} Ora passo la parola alla mia collega Valentina, che le farà alcune domande più specifiche riguardo alla tecnologia all’interno dei musei.
    
    \item \textbf{Intervistatore:} Le volevo chiedere se attualmente, nel museo in cui lavora, c’è un sistema digitale o tecnologico per acquistare non solo i biglietti, ma anche audioguide. Se ci sono, quali sono le difficoltà che vengono riscontrate più spesso, soprattutto dai giovani?
    
    \item \textbf{Intervistato:} Sì, è possibile acquistare i biglietti online. Come ho detto, la biglietteria all’interno dei musei in Sicilia è gestita da società private, chiamate “servizi aggiuntivi”. Queste società ottengono l’appalto per gestire la biglietteria, la vendita di libri, di supporti didattici come le audioguide, i servizi didattici e i bar. I supporti didattici al museo sono presenti, come ho accennato prima. Tutto è migliorabile, nel senso che si potrebbe fare sempre di più. Comunque, ci sono video e il biglietto si può acquistare online, anche in forma cumulativa.
    
    \item \textbf{Intervistatore:} Ha qualche suggerimento su come migliorare la situazione?
    
    \item \textbf{Intervistato:} Si potrebbe estendere il biglietto cumulativo a tutti i siti della città, come il Museo Paolo Orsi, il Museo Bellomo, il Castello Maniace, il Castello Eurialo, il Parco Archeologico. Un biglietto unico aiuterebbe il turista, magari valido per quattro o cinque giorni, dato che non si può vedere tutto in un solo giorno. Il problema è che non tutti i siti sono gestiti dalla stessa società, quindi è difficile. Bisognerebbe fare questo lavoro a monte, nel senso che quando si fa la gara d’appalto, dovrebbe essere per tutti i siti, non solo per alcuni. Così si potrebbe creare un biglietto unico per tutti i siti della città.
    
    \item \textbf{Intervistatore:} Attualmente, al museo ci sono attività di realtà virtuale o aumentata che possono integrare l’esperienza?
    
    \item \textbf{Intervistato:} Quando si organizzano mostre, sì. Ad esempio, attualmente al Museo Archeologico di Siracusa c’è una bellissima mostra sui Micenei nella sala rotonda al centro della pianta. Non so se avete mai visitato il Museo Paolo Orsi, ma avrete notato tutti i supporti e i video presenti. Per quanto riguarda la realtà aumentata, sarebbe bello implementare dispositivi come visori che ti fanno sentire di essere all’interno di un ambiente storico. Sarebbe fantastico poter vedere in 3D il Teatro Greco di Siracusa com’era 2.500 anni fa, o vedere come si costruiva un vaso greco nel V secolo a.C. Anche al Parco Archeologico sarebbe bellissimo avere la realtà aumentata del teatro o vedere Ortigia nel VII-VI secolo a.C. con queste tecnologie.
    
    \item \textbf{Intervistatore:} È su questo che volevamo indagare per capire cosa si potrebbe fare.
    
    \item \textbf{Intervistato:} Dovreste parlarne con il direttore del museo per vedere se ci sono progetti o idee in merito.
    
    \item \textbf{Intervistatore:} Certo, se dovessimo realizzarle, prenderemo sicuramente contatti. Un’ultima domanda: sa se il museo utilizza i social media o app per promuovere le sue attività?
    
    \item \textbf{Intervistato:} Sì, c’è una pagina Facebook dove vengono pubblicate tutte le novità, i prezzi, le attività e i convegni che organizza il museo. Ad esempio, domani c’è un convegno al museo, il 10, 11 e 12. Sabato 12 si inaugura un nuovo settore del Museo Archeologico di Siracusa.
    
    \item \textbf{Intervistatore:} Purtroppo in questo momento siamo a Milano, ma promuoveremo l’iniziativa ai nostri conoscenti a Siracusa.
    
    \item \textbf{Intervistato:} Al Parco Archeologico, alla Villa del Tellaro a Palazzolo, c’è questa iniziativa in programma.
    
    \item \textbf{Intervistatore:} L’ultima domanda: in che modo, secondo lei, l’esperienza museale e la cultura influiscono sul benessere mentale e sociale degli studenti? In che modo li può arricchire?
    
    \item \textbf{Intervistato:} È fondamentale conoscere la storia della propria città fin dall’infanzia; è la cosa più bella di questo mondo. Più cose si sanno, meno cose cattive si fanno, perché, come diceva Socrate, la madre di ogni male è l’ignoranza. La conoscenza non ha mai portato male, ha sempre portato bene. Più conosci, più bene fai. Questo è senz’altro il mio pensiero.
    
    \item \textbf{Intervistatore:} Quindi secondo lei un ricco bagaglio culturale può aiutare a far sentire le persone parte di una comunità, renderle cittadini più attivi?
    
    \item \textbf{Intervistato:} Questa è una cosa che manca, ad esempio, a Siracusa. In Sicilia, in generale, pochissimi conoscono la storia della propria città, e questo si dovrebbe fare fin dall’infanzia, fin dalle scuole elementari.
    
    \item \textbf{Intervistatore:} Sarebbe bello promuovere ancora di più i rapporti tra musei e scuole, magari studiando prima in classe ciò che si andrà a vedere, così i ragazzi sono più interessati.
    
    \item \textbf{Intervistato:} Le scuole vengono, e ne vengono a centinaia. Però quello che manca è la preparazione da parte degli insegnanti. Una cosa è venire per conoscere, un’altra è venire per passare il tempo, per passeggiare, per trascorrere una giornata al sole in gita. Questo è il problema principale.
    
    \item \textbf{Intervistatore:} La ringraziamo molto per il suo tempo. È stato davvero importante per noi fare questa intervista e ricevere il suo punto di vista. Le auguriamo una buona giornata.
    
    \item \textbf{Intervistato:} Grazie a voi, arrivederci.
\end{itemize}

\end{document}