\documentclass{article}
\usepackage[utf8]{inputenc}
\usepackage[italian]{babel}
\usepackage{graphicx}
\usepackage{hyperref}
\usepackage{pgf-pie} % Pacchetto per grafici a torta
\usepackage{lscape} % Per pagine orizzontali
\usepackage{caption} % Per personalizzare le didascalie
\usepackage[a4paper, margin=2.5cm]{geometry}

\title{Deadline 1 – Needfinding}
\author{Mattia Colombo, Carmen Giaccotto, Alessia Franchetti-Rosada \\Federico Previtali, Manoueil Michael Halim Riad Hanna \\ Valentina Petrignano, Michele Arrigoni}
\date{14 Ottobre 2024}

\begin{document}

\maketitle

\section{Introduzione}

\subsection{Dominio di interesse e perché lo abbiamo scelto}

Il nostro team è composto da: Mattia Colombo, Carmen Giaccotto, Alessia Franchetti-Rosada, Federico Previtali, Manoueil Michael Halim Riad Hanna, Valentina Petrignano e Michele Arrigoni.
Il nome del gruppo, Designer For Culture, è stato scelto sin da subito in quanto la nostra missione consiste nell’esplorare e sviluppare soluzioni innovative che uniscano tecnologia e cultura, con l’obiettivo di rendere l’arte e il patrimonio culturale più accessibili e coinvolgenti e promuovere una cittadinanza più consapevole e partecipativa. Abbiamo deciso di concentrarci su un target giovane, composto da ragazzi fra i 16 e i 19 anni che frequentano licei o che hanno da poco iniziato l’università. Il nostro intento iniziale è quello di capire che tipo di rapporto hanno gli adolescenti con l’arte e la cultura, in particolare quella delle loro città di appartenenza e capire quali sono i lori bisogni e le loro richieste per rendere più interattiva e coinvolgente l’esperienza museale.

\subsection{Partecipanti}

I partecipanti sono stati selezionati tramite conoscenze personali, tramite l’uso di piattaforme social come LinkedIn oppure tramite i contatti lasciati all’interno del survey da alcuni utenti.
Gli adolescenti, in particolare, sono stati coinvolti per comprendere le loro esigenze e motivazioni riguardo alle esperienze museali.
Come esperti di dominio abbiamo intervistato persone che lavorano nei musei da anni e che hanno esperienza adatta per darci consigli su cosa andrebbe e su cosa non andrebbe modificato all’interno dei musei. Abbiamo inoltre intervistato i genitori degli utenti rappresentativi, per vedere se la loro visione fosse la stessa dei figli e se fossero in grado di darci ulteriori informazioni da un punto di vista esterno, ma vicino a quello dei ragazzi.\\

\textbf{Utenti rappresentativi}: ragazzi tra i 16 e 19 anni.
\begin{itemize}
\item Aurora: Femmina, 19 anni, studentessa universitaria.
\item 	Daniele: Maschio, 19 anni, studente universitario.
\item 	Marco: Maschio, 18 anni, studente del quinto anno di Liceo Scientifico.
\item 	Ludovica: Femmina, 16 anni, studentessa del terzo anno di Liceo Scientifico.
\end{itemize}

\textbf{Esperti di Dominio}: persone con esperienza nel settore museale.
\begin{itemize}
\item Camilla: femmina, Education Officer presso il Museo della Scienza e della Tecnologia di Milano.
\item Carlo: maschio, Presidente dell’ArcheoClub di Siracusa e membro del consiglio direttivo dell’Associazione Nazionale Guide Turistiche.
\end{itemize}

\textbf{Utenti Affini}: genitori degli utenti rappresentativi.
\begin{itemize}
\item Danila: femmina, madre di due ragazzi, rispettivamente di 15 e 18 anni.
\item 	Paolo: maschio, padre di Marco (18 anni).
\item 	Stefania: femmina, madre di Ludovica (16 anni).
\end{itemize}

\subsection{Luogo delle Interviste}

Sulla base delle disponibilità dei partecipanti le interviste sono state svolte principalmente a distanza tramite un piattaforma di video conferenze.

\subsection{Organizzazione dell'Intervista}
Le interviste sono state organizzate in modo da raccogliere informazioni nel modo più completo possibile, in particolare si sono fatte considerazioni specifiche per:

\begin{itemize}
    \item I \textbf{gestori museali}, sfruttando il punto di vista di chi vive a contatto diretto con la categoria che ci interessa coinvolgere nel progetto. Sono state fatte domande focalizzate ad indagare come gli adolescenti interagiscono con l'arte e i musei, le sfide da affrontare nel coinvolgerli e quali potrebbero essere le migliori strategie per farlo.
    \item Gli \textbf{adolescenti}, indagando su quali siano le loro attuali esperienze e interazioni con i musei, le loro preferenze e quali sono le barriere, gli ostacoli o le motivazioni che condizionano la loro esperienza.
    \item I \textbf{genitori degli utenti rappresentativi}, per confrontare le loro opinioni con quelle dei figli e ottenere ulteriori informazioni da una prospettiva esterna, ma comunque vicina a quella dei giovani.
\end{itemize}

\subsection{Ruolo dei Membri del Gruppo}

Tutti i membri del gruppo hanno partecipato all'organizzazione delle interviste o alla loro realizzazione. Non vi sono stati ruoli fissi per tutte le interviste rendendo tutti i membri più o meno partecipi in questo aspetto del progetto.

\subsection{Materiale Usato}

Durante le interviste sono stati utilizzati:

\begin{itemize}
    \item Registratore audio (previo consenso dei partecipanti) per garantire l'accuratezza dei dati raccolti.
    \item Notebook per prendere appunti durante le interviste.
    \item Software di videoconferenza.
\end{itemize}

\section{Risultati delle Interviste}
Partendo dalle interviste raccolte, abbiamo individuato i bisogni degli adolescenti nel contesto museale. Questi bisogni emergono dalle esperienze, opinioni e suggerimenti condivisi dagli intervistati, sia dagli adolescenti stessi che dai professionisti del settore museale.

\subsection{Bisogni degli Utenti e Citazioni Chiave Rilasciate}
\begin{enumerate}
    \item \textbf{Necessità di esperienze interattive e coinvolgenti}
    \begin{itemize}
        \item Gli adolescenti hanno bisogno di un modo per interagire attivamente con il museo attraverso attività che includano interazione, manualità o utilizzo di tecnologie per stimolare il loro interesse.
        \item \textbf{Riferimenti:}
        \begin{itemize}
            \item \textbf{Camilla}: ``I laboratori che includono elementi di interazione, manualità o l'uso di tecnologie sono molto più stimolanti per loro.''
            \item \textbf{Aurora}: ``Alla mostra di Van Gogh c'era una sala immersiva... Era una parte più interattiva rispetto al resto del percorso.''
            \item \textbf{Ludovica}: ``Mi piacciono le installazioni interattive. Penso che i musei dovrebbero offrire più esperienze interattive, specialmente per i giovani.''
            \item \textbf{Marco}: ``Apprezzo i musei in cui il visitatore può fare qualcosa, quelli interattivi.''
        \end{itemize}
    \end{itemize}

    \item \textbf{Necessità di esperienze immersive}
    \begin{itemize}
        \item Gli adolescenti hanno bisogno di poter immergersi nell'esperienza museale attraverso installazioni interattive o esperienze digitali immersive.
        \item \textbf{Riferimenti:}
        \begin{itemize}
            \item \textbf{Camilla}: ``Il nostro programma \emph{Digital Aesthetics} offre installazioni di arte digitale, immersiva e interattiva. Queste esperienze sono molto apprezzate dai ragazzi.''
            \item \textbf{Aurora}: ``Alla Sagrada Familia potevi vedere da vicino dettagli non visibili a occhio nudo grazie alla tecnologia.''
            \item \textbf{Ludovica}: ``Installazioni interattive dove puoi entrare nelle opere.''
        \end{itemize}
    \end{itemize}

    \item \textbf{Necessità di esplorazione individuale}
    \begin{itemize}
        \item Gli adolescenti hanno bisogno di un modo per esplorare e scoprire i musei in autonomia, piuttosto che attraverso visite di gruppo tradizionali.
        \item \textbf{Riferimenti:}
        \begin{itemize}
            \item \textbf{Aurora}: ``Trovo un po' noiose le visite di gruppo. Sarebbe meglio focalizzarsi su aspetti che permettano a ciascuno di avere un'esperienza completa anche girando autonomamente.''
            \item \textbf{Ludovica}: ``Preferisco scoprire le cose da sola o con qualcuno che conosca bene l'argomento e che possa spiegarmelo in modo più dinamico.''
        \end{itemize}
    \end{itemize}

    \item \textbf{Necessità di spazi sociali all'interno del museo}
    \begin{itemize}
        \item Gli adolescenti hanno bisogno che i musei diventino spazi sociali dove poter incontrarsi, discutere d'arte e sentirsi parte di una comunità.
        \item \textbf{Riferimenti:}
        \begin{itemize}
            \item \textbf{Ludovica}: ``Mi piacerebbe che ci fossero spazi dove i ragazzi possono incontrarsi e parlare d'arte.''
        \end{itemize}
    \end{itemize}

    \item \textbf{Necessità di un uso appropriato della tecnologia}
    \begin{itemize}
        \item Gli adolescenti hanno bisogno che la tecnologia sia utilizzata in modo appropriato nei musei, migliorando l'esperienza senza essere eccessiva o fuori contesto.
        \item \textbf{Riferimenti:}
        \begin{itemize}
            \item \textbf{Daniele}: ``La tecnologia deve essere usata in contesti e modi appropriati. Non vorrei vedere i quadri di Van Gogh scorrere sotto i miei piedi su uno schermo OLED; mi sembrerebbe inutile.''
            \item \textbf{Ludovica}: ``Non mi piace molto usare il telefono mentre sono al museo, preferisco godermi l'esperienza dal vivo.''
        \end{itemize}
    \end{itemize}

    \item \textbf{Necessità di partecipare attivamente e creare}
    \begin{itemize}
        \item Gli adolescenti hanno bisogno di opportunità per partecipare attivamente e creare qualcosa all'interno del museo, come workshop o laboratori.
        \item \textbf{Riferimenti:}
        \begin{itemize}
            \item \textbf{Ludovica}: ``Una parte del museo potrebbe essere dedicata alla creazione, dove i visitatori possono partecipare e creare qualcosa.''
            \item \textbf{Camilla}: ``I laboratori che combinano tecnologia e manualità permettono ai ragazzi di apprendere attraverso il fare.''
        \end{itemize}
    \end{itemize}

    \item \textbf{Necessità di apprendere attraverso metodi coinvolgenti}
    \begin{itemize}
        \item Gli adolescenti hanno bisogno di modi coinvolgenti per apprendere sulle opere e sugli autori, ad esempio attraverso video o storytelling.
        \item \textbf{Riferimenti:}
        \begin{itemize}
            \item \textbf{Marco}: ``Quando ci sono video che raccontano l'opera o la vita dell'autore, magari recitati da attori, è una cosa che apprezzo molto.''
        \end{itemize}
    \end{itemize}

    \item \textbf{Necessità di promozione attraverso i social media}
    \begin{itemize}
        \item Gli adolescenti hanno bisogno che i musei utilizzino maggiormente i social media per promuovere le attività e coinvolgere i giovani.
        \item \textbf{Riferimenti:}
        \begin{itemize}
            \item \textbf{Stefania}: ``Penso che i social media siano il mezzo principale. I musei dovrebbero sfruttare di più piattaforme come Instagram e TikTok.''
            \item \textbf{Daniele}: Ha suggerito un sistema di recensioni e un'app dedicata.
        \end{itemize}
    \end{itemize}

\end{enumerate}

\section{Risultati del Survey}

Abbiamo distribuito un sondaggio per raccogliere dati preliminari sugli utenti, in particolare adolescenti e giovani adulti. I risultati ci hanno fornito un quadro utile delle loro esperienze con i musei e delle loro aspettative.

\subsection{Link al questionario}

Il questionario completo può essere consultato al seguente link: \url{https://bit.ly/4eu3ox7}

\subsection{Demografia dei Partecipanti}

Il sondaggio ha coinvolto principalmente partecipanti di età compresa tra i 16-19 anni e oltre i 20 anni, con una rappresentanza femminile predominante. La maggior parte dei partecipanti risiede in piccole città o paesi, sebbene ci siano stati anche rispondenti provenienti da aree urbane.

\subsection{Frequenza di Visita ai Musei}

Quando è stato chiesto quante volte avevano visitato un museo negli ultimi 12 mesi, la maggior parte dei partecipanti ha indicato tra le 3 e le 5 visite, con alcuni che hanno dichiarato di aver visitato musei più di 5 volte.

\subsection{Grafici a Torta per i Dati del Survey}

\begin{figure}[h]
    \centering
    \begin{tikzpicture}
        \pie[text=legend, radius=3, color={red!30, blue!30, green!30, yellow!30}]{
            40/16-19 anni,
            35/20-25 anni,
            25/Oltre 25 anni
        }
    \end{tikzpicture}
    \caption{Distribuzione dell'età dei partecipanti.}
\end{figure}

\begin{figure}[h]
    \centering
    \begin{tikzpicture}
        \pie[text=legend, radius=3, color={red!50, blue!50}]{
            60/3-5 visite,
            40/Più di 5 visite
        }
    \end{tikzpicture}
    \caption{Frequenza di visita ai musei negli ultimi 12 mesi.}
\end{figure}
\newpage

\subsection{Suggerimenti per il Miglioramento dell'Esperienza Museale}

\begin{enumerate}
    \item \textbf{Migliorare l'interattività e il coinvolgimento}
        \begin{itemize}
            \item \textit{Mancanza di contenuti interattivi}: Un numero significativo di partecipanti ha indicato la ``mancanza di contenuti interattivi o interessanti'' come una delle principali difficoltà nell'esperienza museale.
            \item \textit{Interesse per attività interattive e giochi}: Molti adolescenti hanno espresso il desiderio di partecipare ad attività interattive e giochi all'interno del museo, indicando che tali attività li invoglierebbero a visitare più spesso.
            \item \textit{Preferenze per esposizioni interattive}: Le ``esposizioni interattive'' sono state citate come uno dei motivi principali che attirano i giovani al museo.
        \end{itemize}

    \item \textbf{Integrare tecnologie immersive}
        \begin{itemize}
            \item \textit{Interesse per VR e AR}: La tecnologia, in particolare la realtà virtuale (VR) e aumentata (AR), è stata menzionata come un elemento attrattivo, con alcuni partecipanti che desiderano più installazioni che utilizzino queste tecnologie.
            \item \textit{Desiderio di tecnologie innovative}: L'uso di installazioni digitali e tecnologie all'avanguardia è stato indicato come un fattore che potrebbe migliorare l'esperienza museale.
        \end{itemize}

    \item \textbf{Facilitare l'esplorazione individuale}
        \begin{itemize}
            \item \textit{Preferenza per l'autonomia}: Molti giovani preferiscono approfondire le informazioni leggendo le descrizioni accanto alle opere o cercando informazioni autonomamente tramite smartphone.
            \item \textit{Interesse per app personalizzate}: C'è un desiderio di avere app che consentano di personalizzare il percorso di visita in base agli interessi individuali.
        \end{itemize}

    \item \textbf{Offrire alternative alle guide fisiche}
        \begin{itemize}
            \item \textit{Interesse per audioguide su smartphone}: Molti partecipanti sarebbero più propensi a utilizzare audioguide se accessibili direttamente dal proprio smartphone con le proprie cuffie.
            \item \textit{Preferenza per dispositivi personali}: C'è una tendenza a preferire l'utilizzo di dispositivi personali per accedere ai contenuti, evitando dispositivi condivisi.
        \end{itemize}
    
    \item \textbf{Creazione di spazi sociali all'interno del museo}
        \begin{itemize}
            \item \textit{Interesse per la connessione sociale}: Alcuni partecipanti sarebbero interessati a un'app che permetta di connettersi con persone con interessi simili durante la visita.
            \item \textit{Desiderio di competizione amichevole}: L'idea di sfide e competizioni con amici o altri visitatori è stata accolta positivamente da diversi adolescenti.
        \end{itemize}

    \item \textbf{Uso appropriato e mirato della tecnologia}
        \begin{itemize}
            \item \textit{Importanza della facilità d'uso}: La ``facilità di navigazione nell'app'' è stata indicata come un elemento importante.
            \item \textit{Evitare sovraccarico tecnologico}: Alcuni partecipanti preferiscono non utilizzare eccessivamente dispositivi personali per non distogliere l'attenzione dalle opere.
        \end{itemize}
        
    \item \textbf{Promozione della partecipazione attiva e creativa}
        \begin{itemize}
            \item \textit{Desiderio di attività pratiche}: Gli adolescenti hanno mostrato interesse per eventi, laboratori tematici e attività che permettano un coinvolgimento attivo.
            \item \textit{Interesse per giochi e sfide}: La partecipazione a cacce al tesoro interattive e giochi competitivi è vista come un modo per rendere la visita più divertente.
        \end{itemize}

    \item \textbf{Apprendimento attraverso metodi coinvolgenti}
        \begin{itemize}
            \item \textit{Preferenza per esperienze interattive}: Le esperienze con materiale audiovisivo e interattivo sono considerate coinvolgenti e stimolanti.
            \item \textit{Interesse per storytelling e curiosità}: Scoprire curiosità e informazioni nascoste sulle opere è un forte incentivo per i giovani visitatori.
        \end{itemize}

    \item \textbf{Promozione attraverso i social media e piattaforme digitali}
        \begin{itemize}
            \item \textit{Utilizzo dei social media}: I giovani suggeriscono di utilizzare piattaforme come Instagram e TikTok per promuovere mostre ed eventi.
            \item \textit{Collaborazioni con influencer}: Coinvolgere figure popolari potrebbe aumentare l'attrattiva dei musei per gli adolescenti.
        \end{itemize}

    \item \textbf{Offrire esperienze autentiche e personalizzate}
        \begin{itemize}
            \item \textit{Personalizzazione del percorso}: La possibilità di personalizzare il percorso di visita è stata apprezzata.
            \item \textit{Desiderio di autenticità}: I giovani vogliono vivere esperienze genuine, senza eccessive mediazioni tecnologiche.
        \end{itemize}

    \item \textbf{Integrazione di visite virtuali a siti antichi o non accessibili}
        \begin{itemize}
            \item \textit{Interesse per tecnologie immersive}: L'uso di VR e AR per esplorare siti storici ha suscitato interesse.
            \item \textit{Desiderio di esperienze uniche}: I giovani sono attratti da esperienze che non possono vivere altrove.
        \end{itemize}

    \item \textbf{Semplificazione dell'accesso e della fruizione dei servizi}
        \begin{itemize}
            \item \textit{Difficoltà con code e procedure}: Le code e le procedure complicate per ottenere biglietti e audioguide sono viste come ostacoli.
            \item \textit{Desiderio di servizi digitali efficienti}: La possibilità di acquistare biglietti e accedere ad audioguide tramite smartphone è altamente apprezzata.
        \end{itemize}

    \item \textbf{Collaborazione con il settore educativo}
        \begin{itemize}
            \item \textit{Integrazione scolastica}: Molti partecipanti vorrebbero che le visite ai musei fossero più integrate nel contesto scolastico.
            \item \textit{Valore educativo}: I giovani riconoscono il potenziale dei musei nell'arricchire la propria formazione e crescita personale.
        \end{itemize}

    \item \textbf{Creazione di contenuti accessibili e inclusivi}
        \begin{itemize}
            \item \textit{Diversi livelli di conoscenza}: C'è la necessità di fornire contenuti adatti sia a esperti che a neofiti.
            \item \textit{Accessibilità linguistica}: L'opzione di avere contenuti in diverse lingue è considerata importante da alcuni partecipanti.
        \end{itemize}

    \item \textbf{Feedback continuo e miglioramento}
        \begin{itemize}
            \item \textit{Desiderio di esprimere opinioni}: I partecipanti hanno fornito numerosi suggerimenti su come migliorare l'esperienza museale.
            \item \textit{Importanza dell'aggiornamento}: I giovani apprezzano quando i musei si adattano alle nuove tendenze e tecnologie.
        \end{itemize}
\end{enumerate}

\subsection{Disponibilità a Partecipare a Ulteriori Interviste}

Abbiamo anche chiesto ai partecipanti se fossero disposti a rispondere a ulteriori domande o a partecipare a interviste individuali. La maggior parte ha preferito non essere contattata, anche se alcuni hanno lasciato i loro contatti per un coinvolgimento futuro.

\section{Logica del form}

\vspace*{\fill}
\begin{center}
    \includegraphics[width=\textwidth]{form_logic.png} % Immagine della logica del form (da inserire)
\end{center}
\vspace*{\fill}

\section{Foto Affinity Diagram Esercitazione}

\vspace*{\fill}
\begin{center}
    \includegraphics[width=\textwidth]{affinity_diagram.png} % Spazio per foto Affinity Diagram
\end{center}
\vspace*{\fill}

\section{Sintesi}

\subsection{Mappa Iniziale tra Utenti e Obiettivi}
\vspace*{\fill}
\begin{center}
    \includegraphics[width=\textwidth]{map.png}
\end{center}
\vspace*{\fill}

\section{Passi Futuri}

\begin{itemize}
    \item Analizzare ancora più in dettaglio i dati raccolti per identificare requisiti specifici.
    \item Sviluppare prototipi di soluzioni tecnologiche, interattive e coinvolgenti, basate sui bisogni emersi.
    \item Condurre test di usabilità con utenti rappresentativi per valutare le soluzioni proposte.
    \item Collaborare con musei e istituzioni per implementare e affinare le soluzioni.
\end{itemize}
\end{document}
