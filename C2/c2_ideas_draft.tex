\documentclass{article}
\usepackage[utf8]{inputenc}
\usepackage[italian]{babel}
\usepackage{graphicx}
\usepackage{hyperref}
\usepackage{pgf-pie} % Pacchetto per grafici a torta
\usepackage{lscape} % Per pagine orizzontali
\usepackage{caption} % Per personalizzare le didascalie
\usepackage[a4paper, margin=2.5cm]{geometry}
\usepackage{enumitem} % Per liste personalizzate
\definecolor{colore1}{HTML}{725A4A}
\definecolor{colore2}{HTML}{472C19}
\definecolor{colore3}{HTML}{A46F4C}
\definecolor{colore4}{HTML}{C8C4BD}

\title{Consegna 2 – Domande per il Focus Group e Idee per le Contextual Inquiries}
\author{Mattia Colombo, Carmen Giaccotto, Alessia Franchetti-Rosada \\Federico Previtali, Manoueil Michael Halim Riad Hanna \\ Valentina Petrignano, Michele Arrigoni}
\date{\today}

\begin{document}

\maketitle

\section*{Introduzione}

Per la \textbf{Consegna 2}, abbiamo preparato una serie di domande per il \textbf{focus group} e sviluppato idee per le \textbf{contextual inquiries}, al fine di approfondire i bisogni degli utenti e raccogliere ulteriori informazioni per il nostro progetto. Le domande del focus group sono state ideate per stimolare la discussione tra i partecipanti, mentre le idee per le contextual inquiries ci aiuteranno a osservare e comprendere il comportamento degli utenti in situazioni reali.

\section{Domande per il Focus Group}

\subsection{Spazi sociali e di condivisione nei musei}

\begin{itemize}
    \item Che tipo di spazi sociali vorreste vedere nei musei?
    \item Vi piacerebbe avere aree dedicate a discussioni informali o luoghi per rilassarvi e socializzare con gli amici?
    \item Come immaginate un’area perfetta per parlare d’arte o condividere impressioni con gli altri visitatori?
\end{itemize}

\subsection{Esperienze interattive e immersive}

\begin{itemize}
    \item Quali tipi di esperienze interattive vi hanno coinvolto maggiormente in passato?
    \item Se un museo offrisse attività pratiche o digitali, cosa vi piacerebbe provare? (es.\ laboratori, installazioni digitali, simulazioni AR/VR)
    \item Le esperienze immersive che sfruttano proiezioni o ambientazioni digitali migliorano il vostro coinvolgimento?
\end{itemize}

\subsection{Uso della tecnologia nei musei}

\begin{itemize}
    \item Come valutate l’uso della tecnologia nei musei? Preferite tecnologie come audioguide o app per esplorare i contenuti in modo autonomo?
    \item La tecnologia ha mai reso l’esperienza museale meno autentica per voi?
    \item Come vorreste che la tecnologia fosse integrata nell’esperienza museale?
\end{itemize}

\subsection{Apprendimento coinvolgente}

\begin{itemize}
    \item Qual è la vostra modalità preferita per apprendere in un museo? Preferite video, narrazioni interattive, giochi educativi o altre attività?
    \item Quanto è importante per voi poter esplorare autonomamente le informazioni in un museo?
    \item Cosa rende un museo più stimolante dal punto di vista educativo?
\end{itemize}

\subsection{Personalizzazione dell’esperienza e percorsi meno battuti}

\begin{itemize}
    \item Vi piacerebbe poter personalizzare i percorsi della vostra visita al museo? Come?
    \item Pensate che l’idea di esplorare percorsi meno conosciuti tramite un’app sia interessante? Quali elementi interattivi sarebbero più motivanti?
\end{itemize}

\subsection{Promozione attraverso i social media e piattaforme digitali}

\subsubsection{Utilizzo dei social media}

\begin{itemize}
    \item In che modo i social media influenzano la tua decisione di visitare un museo?
    \item Ti piacerebbe che i musei fossero più attivi su piattaforme come Instagram o TikTok? Come potrebbero utilizzarle per coinvolgerti di più?
    \item Hai mai condiviso le tue esperienze museali sui social media? Cosa ti spinge a farlo o a non farlo?
\end{itemize}

\subsubsection{Collaborazioni con influencer}

\begin{itemize}
    \item Pensi che la collaborazione tra musei e influencer possa rendere l’arte più accessibile e interessante per i giovani?
    \item Segui influencer o blogger che parlano di arte o musei? Come influenzano il tuo interesse verso le visite museali?
\end{itemize}

\subsection{Integrazione di visite virtuali a siti antichi o non accessibili}

\subsubsection{Interesse per tecnologie immersive}

\begin{itemize}
    \item Saresti interessato a visitare virtualmente siti storici o mostre non accessibili fisicamente? Perché?
    \item Come valuti l’utilizzo di realtà virtuale (VR) o aumentata (AR) per esplorare luoghi storici o opere d’arte?
    \item Pensi che le visite virtuali possano arricchire o sostituire l’esperienza reale del museo? In che modo?
\end{itemize}

\subsection{Semplificazione dell’accesso e della fruizione dei servizi}

\subsubsection{Difficoltà con code e procedure}

\begin{itemize}
    \item Hai mai riscontrato problemi o lunghe attese nell’acquisto di biglietti per un museo? Raccontami la tua esperienza.
    \item Quali aspetti delle procedure di accesso al museo ti creano maggiore frustrazione o disagio?
\end{itemize}

\subsubsection{Desiderio di servizi digitali ed efficienti}

\begin{itemize}
    \item Ti piacerebbe poter acquistare i biglietti online o tramite un’applicazione mobile? Quanto sarebbe utile per te?
    \item Come potrebbe la tecnologia semplificare la tua esperienza museale, dall’acquisto del biglietto alla visita?
    \item Apprezzeresti la possibilità di evitare code grazie a servizi digitali?
\end{itemize}

\subsection{Collaborazione con il settore educativo}

\subsubsection{Integrazione scolastica}

\begin{itemize}
    \item Quanto ritieni utili le visite ai musei organizzate dalla scuola? Cosa apprezzi e cosa cambieresti?
    \item In che modo i musei potrebbero collaborare meglio con le scuole per rendere le visite più interessanti o rilevanti per te?
    \item Ti piacerebbe partecipare a progetti o attività museali legate ai tuoi studi? Quali tipi di attività potrebbero coinvolgerti di più?
\end{itemize}

\subsubsection{Valore educativo}

\begin{itemize}
    \item Quali aspetti delle visite museali contribuiscono maggiormente alla tua formazione personale o scolastica?
    \item Ci sono argomenti o temi che vorresti approfondire attraverso le attività museali?
\end{itemize}

\subsection{Creazione di contenuti accessibili e inclusivi}

\subsubsection{Diversi livelli di conoscenza}

\begin{itemize}
    \item Preferisci che le informazioni sulle opere siano dettagliate o sintetiche? Perché?
    \item Ti piacerebbe avere la possibilità di scegliere il livello di approfondimento delle spiegazioni (es.\ principianti, intermedi, avanzati)?
    \item Come potremmo rendere i contenuti museali più adatti ai tuoi interessi e al tuo livello di conoscenza?
\end{itemize}

\subsubsection{Accessibilità linguistica}

\begin{itemize}
    \item Quanto è importante per te che le informazioni siano disponibili nella tua lingua madre o in altre lingue che conosci?
    \item Hai mai avuto difficoltà a comprendere le spiegazioni a causa del linguaggio utilizzato o della mancanza di traduzioni?
    \item Come potremmo migliorare l’accessibilità linguistica dei contenuti museali?
\end{itemize}

\section{Idee per le Contextual Inquiries}

\subsection{Osservazione durante una visita museale}

\textbf{Descrizione:}

Accompagnare un piccolo gruppo di adolescenti durante una visita al museo. Prendere nota di come interagiscono con le esposizioni, se utilizzano app o strumenti digitali e quali aree li attirano di più. Chiedere loro di descrivere i propri pensieri in tempo reale per capire le reazioni immediate.

\subsection{Partecipazione a un laboratorio interattivo}

\textbf{Descrizione:}

Organizzare o partecipare a un laboratorio pratico o interattivo all’interno di un museo. Osservare come gli adolescenti reagiscono alle attività manuali o digitali, se sono coinvolti attivamente e se ci sono momenti di frustrazione o interesse particolare.

\subsection{Osservazione di momenti di pausa e spazi sociali}

\textbf{Descrizione:}

Durante una visita, osservare come i giovani utilizzano gli spazi di pausa (come aree ristoro o lounge). Verificare se usano questi spazi per socializzare, rilassarsi o discutere delle mostre e se c’è un desiderio di ulteriori interazioni sociali.

\subsection{Uso di tecnologie di supporto alla visita}

\textbf{Descrizione:}

Assegnare ai partecipanti strumenti come audioguide su smartphone o app con percorsi personalizzati. Osservare come utilizzano questi strumenti, se trovano facilmente le informazioni e se la tecnologia migliora o complica l’esperienza.

\subsection{Osservazione dell’utilizzo dei social media durante la visita al museo}

\textbf{Descrizione:}

Durante una visita museale, accompagnare un gruppo di adolescenti e osservare come e quando utilizzano i social media. Prendere nota se scattano foto, registrano video o condividono esperienze in tempo reale. Chiedere loro quali motivazioni li spingono a condividere o meno sui social e quali caratteristiche del museo favoriscono questa condivisione.

\subsection{Esperienza di una visita virtuale a un sito non accessibile}

\textbf{Descrizione:}

Organizzare una sessione in cui i partecipanti utilizzano dispositivi di realtà virtuale (VR) o aumentata (AR) per visitare virtualmente un sito storico o una mostra non accessibile fisicamente. Osservare le loro reazioni, il livello di coinvolgimento e raccogliere feedback su come questa esperienza si confronta con una visita reale. Chiedere quali aspetti hanno apprezzato di più e quali miglioramenti suggerirebbero.

\subsection{Simulazione del processo di acquisto e accesso ai servizi del museo}

\textbf{Descrizione:}

Chiedere ai partecipanti di simulare l’acquisto di un biglietto per il museo, sia online che in loco. Osservare le difficoltà incontrate, il tempo impiegato e raccogliere le loro impressioni su come migliorare l’efficienza e la semplicità delle procedure di accesso. Valutare anche l’utilizzo di eventuali app o servizi digitali offerti dal museo.

\subsection{Partecipazione a un progetto educativo collaborativo tra museo e scuola}

\textbf{Descrizione:}

Collaborare con una scuola per organizzare una visita museale integrata nel percorso educativo degli studenti. Osservare come i ragazzi interagiscono con le attività proposte, il loro livello di interesse e raccogliere feedback su come la collaborazione tra museo e scuola possa essere migliorata per arricchire l’esperienza formativa.

\subsection{Valutazione dell’accessibilità e dell’inclusività dei contenuti museali}

\textbf{Descrizione:}

Durante una visita al museo, accompagnare un gruppo di adolescenti con diversi livelli di conoscenza dell’arte. Osservare come interagiscono con le informazioni fornite, se trovano le spiegazioni adeguate al loro livello e se avrebbero bisogno di contenuti più accessibili o in diverse lingue. Raccogliere le loro opinioni su come i musei potrebbero rendere i contenuti più inclusivi e comprensibili per tutti.



\end{document}