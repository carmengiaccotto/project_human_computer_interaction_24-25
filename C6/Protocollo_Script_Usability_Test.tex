\documentclass{article}
\usepackage[utf8]{inputenc}
\usepackage{xcolor}
\usepackage{graphicx} 

% Definisci il colore per le sezioni
\definecolor{subsectioncolor}{RGB}{0, 102, 204}

% Impostazioni per la pagina
\usepackage{geometry}
\geometry{a4paper, margin=1in}

% Pacchetto per hyperlink
\usepackage{hyperref}
\hypersetup{
    colorlinks=true,
    linkcolor=blue,
    urlcolor=blue,
    citecolor=blue
}

\begin{document}
\begin{center}
    \includegraphics[width=0.7\textwidth]{logo.png} \\[1em]
    {\LARGE \textbf{ArtMyWay - DesignersForCulture}} \\[0.5em]
    {\Large \textbf{PROTOCOLLO E SCRIPT USABILITY TEST}} \\[1.5em]
    {\large 27 Dicembre 2024}
\end{center}

\section{Partecipanti Coinvolti}
Prima di condurre i test sulla versione high-fidelity del prototipo, sono stati selezionati 7 adolescenti di et\`a compresa tra i 16 e i 19 anni, in linea con il target definito nella fase iniziale del progetto. \\ La selezione tiene conto di due aspetti chiave, ritenuti fondamentali per una valutazione efficace del prototipo:
\begin{itemize}
    \item Esperienza pregressa nella visita di musei;
    \item Familiarit\`a con le applicazioni mobili.
\end{itemize}
Questi criteri permettono di assicurare che i partecipanti rappresentino adeguatamente il pubblico target, fornendo feedback utili e mirati per ottimizzare l'applicazione.

\section{Ruoli dei membri del gruppo}
Ogni caso di test viene eseguito con il contributo di quattro componenti del gruppo, selezionati a turno tra i sette membri complessivi. I ruoli di facilitatore e osservatore sono stati assegnati a priori e rimangono invariati per l'intera durata delle sessioni di testing. Per ciascun test, due facilitatori e due osservatori partecipano attivamente, mentre gli altri componenti del gruppo attendono il proprio turno per i test successivi. Questo approccio garantisce che ogni membro contribuisca nel proprio ruolo, mantenendo coerenza e continuità nell'organizzazione e nella conduzione delle sessioni.

\begin{table}[h!]
    \centering
    \begin{tabular}{|l|l|p{7cm}|}
        \hline
        \textbf{Ruolo} & \textbf{Componente del gruppo} & \textbf{Responsabilit\`a} \\
        \hline
        Facilitatore & Carmen Giaccotto, Alessia Franchetti-Rosada, & Chiede ai partecipanti di eseguire dei task \\ & Mattia Colombo. & utilizzando una o pi\`u interfacce utente.\\
        \hline
        Osservatore & Federico Previtali, Manoueil Michael Halim Riad & Osserva il comportamento del partecipante \\ & Hanna, Valentina Petrignano, Michele Arrigoni. & e ascolta eventuali commenti.\\
        \hline
    \end{tabular}
\end{table}

\section{Materiali e Strumenti necessari per il Test}
\subsection*{Strumenti}
\begin{itemize}
    \item Uno smartphone e/o un computer per la compilazione dei questionari;
    \item Un’applicazione di meeting online per eseguire il test a distanza e registrare l’interazione dell’utente;
    \item Carta e penna per prendere appunti.
\end{itemize}

\subsection*{Materiali}
\begin{itemize}
    \item Modulo di consenso informato;
    \item Questionario pre-test per raccogliere informazioni iniziali;
    \item Questionario post-test per raccogliere impressioni complessive e feedback finali.
\end{itemize}

\section{Esecuzione dello Usability Test}
Il test è condotto individualmente per ciascun utente, seguendo una serie di passaggi ben definiti:

\begin{enumerate}
    \item L'utente firma il modulo di consenso informato.
    \item L'utente compila il questionario \textit{pre-test} per fornire dati anagrafici e altre informazioni generali.
    \item Si raccolgono le idee iniziali degli utenti sulle loro prime impressioni.
    \item Gli utenti eseguono i task richiesti dai facilitatori, mentre gli osservatori annotano le loro reazioni. A ogni partecipante viene chiesto di pensare ad alta voce, seguendo il metodo \textbf{THINK-ALOUD}.
    \item Al termine dei task, vengono richieste ulteriori impressioni ai partecipanti.
    \item Ogni utente compila il questionario \textit{post-test}, basato sul modello SUS (System Usability Scale).
\end{enumerate}

\section{Questionario Pre-Test}
\subsection*{Domande}
\begin{enumerate}
    \item Come ti chiami?
    \item Quanti anni hai?
    \item Qual \`e l'ultimo titolo di studio che hai conseguito?
    \item Come giudichi la tua esperienza d’uso con le app mobili?
    \item Come giudichi la tua esperienza d'uso con i social media?
    \item In media, quante ore alla settimana usi app mobili o tecnologie digitali?
    \item Hai mai utilizzato app per la visita di musei?
    \item Hai mai visitato un museo negli ultimi 12 mesi?
    \item Durante una visita al museo, quali strumenti utilizzi di solito?
\end{enumerate}

\noindent Link al questionario: \url{https://g9dzinv68sa.typeform.com/to/mfIKDzUR}

\section{Analisi dell’esecuzione dei task}
Durante il test viene chiesto ai partecipanti di risolvere i task fondamentali riguardanti il progetto.\\
Per ogni task si misura:
\begin{itemize}
    \item Il tempo di esecuzione;
    \item Il numero di errori commessi;
    \item Il numero di volte in cui \`e necessario l’intervento del facilitatore.
\end{itemize}
Gli errori vengono distinti in due categorie:
\begin{itemize}
    \item \textbf{Moderati}: errori minori che non compromettono in modo significativo l'esecuzione del task, ma possono causare piccoli ritardi o confusione temporanea. 
    \item \textbf{Gravi}: errori che impediscono all'utente di completare il task o di individuare le informazioni necessarie.
\end{itemize}

\newpage

\section{Task Assegnati}
\subsection*{Task Semplice}
"Ricerca e prenotazione di un museo da visitare in base alle proprie preferenze e all’affluenza corrente".
\begin{itemize}
    \item Ricerca di un museo;
    \item Utilizzo dei filtri di ricerca;
    \item Consultazione dell’affluenza del museo;
    \item Prenotazione e acquisto del biglietto;
    \item Ricerca del biglietto acquistato.
\end{itemize}

\subsection*{Task Moderato}
"Personalizzazione della visita con audioguide (in base a preferenze e tempo disponibile) e utilizzo di Virtual Reality".
\begin{itemize}
    \item Personalizzazione del percorso;
    \item Avvio del percorso;
    \item Accesso all'audioguida e alla realt\`a virtuale;
    \item Modifica del percorso.
\end{itemize}

\subsection*{Task Complesso}
"Aggiunta delle opere a una galleria personale digitale, visualizzazione del modello 3D e condivisione con altri utenti".
\begin{itemize}
    \item Accesso alla galleria;
    \item Creazione di una nuova galleria;
    \item Aggiunta/rimozione opere;
    \item Visualizzazione del modello 3D delle opere;
    \item Condivisione del percorso;
    \item Visualizzazione delle condivisioni degli altri utenti;
    \item Scrivere una recensione al museo.
\end{itemize}

\section{Questionario Post-Test}
Il questionario post-test \`e basato sul modello \textbf{SUS (System Usability Scale)}. Ad ogni partecipante viene data la possibilit\`a di scegliere un valore da 1 a 5, dove 1 corrisponde a "fortemente in disaccordo" e 5 a "molto d’accordo". I risultati vengono raccolti in maniera anonima.

\subsection*{Domande}
\begin{enumerate}
    \item Penso che mi piacerebbe usare questo sistema frequentemente.
    \item Ho trovato il sistema inutilmente complesso.
    \item Penso che il sistema sia facile da usare.
    \item Penso che avrei bisogno del supporto di una persona tecnica per poter utilizzare questo sistema.
    \item Ho trovato le varie funzioni di questo sistema ben integrate.
    \item Penso ci siano troppe incoerenze nel sistema.
    \item Immagino che la maggior parte delle persone imparerebbe a usare questo sistema molto rapidamente.
    \item Ho trovato il sistema molto pesante da usare.
    \item Mi sono sentito molto sicuro nell'usare il sistema.
    \item Ho avuto bisogno di imparare molte cose prima di poter iniziare a usare questo sistema.
\end{enumerate}

\noindent Link al questionario: \url{https://g9dzinv68sa.typeform.com/to/h58FnN9J}

\section{Script Usability Test}
\subsection{Introduzione}
Grazie per aver accettato di partecipare al test di usabilità della nostra applicazione. Il tuo contributo è fondamentale per aiutarci a migliorare l'applicazione, rendendola più intuitiva e semplice da usare.\\
Prima di iniziare, ti chiediamo gentilmente di compilare un modulo di consenso informato. Questo documento spiega come utilizzeremo i dati raccolti durante il test e garantisce che le tue informazioni saranno trattate in modo riservato. Ricorda che puoi interrompere il test in qualsiasi momento senza fornire spiegazioni.\\
Se hai domande prima di cominciare, non esitare a chiedere.

\subsection{Prima parte: Quiz Anagrafico}
Per iniziare, ti proponiamo un breve quiz anagrafico per raccogliere alcune informazioni su di te e\\ comprendere meglio il tuo background. Sei pronto per iniziare?\\

\textbf{Se l'utente conferma:} Apri il file che ti abbiamo inviato via e-mail e clicca sul link al questionario pre-test. Completato il questionario, informaci per passare alla fase successiva del test.

\subsection{Seconda parte: Test di Usabilità}
\subsubsection{Osservazione iniziale della schermata home}
Ti mostreremo per pochi secondi la schermata principale dell’applicazione. Per il momento, ti chiediamo di non cliccare nulla ma solo di osservare attentamente.\\

\textbf{Dopo 5 secondi:} 
Stop. Vorremmo sapere cosa ne pensi: riesci a intuire a cosa serve l’applicazione e per chi potrebbe essere utile?\\

\textbf{Dopo la risposta:} Ora potrai osservare di nuovo la schermata, questa volta con calma, ma sempre senza interagire.\\

\textbf{Dopo 10 secondi:} 
Quali sono le tue impressioni? Quale punto sceglieresti per iniziare la\\ navigazione? Cosa ti aspetteresti che accada cliccando sui vari elementi?

\subsubsection{Esplorazione guidata del prototipo}
Ora ti guideremo nell'esplorazione del prototipo. Ti chiediamo di pensare ad alta voce mentre interagisci: descrivi cosa fai, cosa noti e cosa pensi. Questo ci aiuterà a capire meglio come utilizzi l’applicazione e quali sono le tue aspettative.

\paragraph*{Task 1: Ricerca e prenotazione di un museo}
Immagina di voler visitare un museo. Puoi mostrarci cosa faresti per cercare un museo da visitare? Tieni a mente che si tratta di un prototipo, quindi alcune funzionalità potrebbero non essere pienamente operative. Ricordati di pensare ad alta voce e chiedi pure se hai dubbi.\\

\textbf{Dopo la ricerca:} 
Hai trovato il museo che cercavi? Vuoi provare un altro metodo per effettuare la ricerca? Sei soddisfatto dei risultati?\\

\textbf{Dopo la prenotazione:} 
Riesci a trovare il biglietto che hai appena acquistato?\\

\textbf{Dopo l'esecuzione:} 
Cosa ne pensi? Hai trovato difficoltà nel completare il compito? Ci sono aspetti che ti sono piaciuti particolarmente o che potrebbero essere migliorati?

\paragraph*{Task 2: Personalizzazione della visita con audioguide}
Ora ti chiediamo di creare un percorso\\ personalizzato per la visita al museo.\\

\textbf{Dopo la creazione del percorso:} 
Riesci ad avviare il percorso che hai appena salvato? Come procederesti se volessi modificarlo?\\

\textbf{Dopo l'esecuzione:} 
Hai trovato difficoltà? C’è qualcosa che ti ha colpito positivamente o negativamente?

\paragraph*{Task 3: Galleria personale e condivisione}
Prova ad utilizzare la galleria personale. Cosa faresti per aggiungere o rimuovere un’opera dalla galleria?\\

Se volessi condividere il tuo percorso o esplorare quelli condivisi da altri utenti, come procederesti?\\

\textbf{Dopo l'esecuzione:} 
Hai riscontrato difficoltà? Ci sono funzionalità che ti hanno particolarmente soddisfatto o che pensi potrebbero essere migliorate?

\subsubsection{Esplorazione libera del prototipo}
Ora puoi esplorare liberamente il prototipo, soffermandoti su funzionalità che ti incuriosiscono o tornando su aspetti già visti per approfondirli. Anche in questa fase, ti invitiamo a pensare ad alta voce per condividere con noi le tue impressioni.

\subsection{Conclusione}
Ora che hai osservato e interagito con il prototipo, ti invitiamo a condividere eventuali impressioni, domande o osservazioni che non hai ancora espresso. Inoltre, ti chiediamo di compilare il questionario post-test. Grazie mille per il tuo tempo e il tuo prezioso contributo. 

\subsection{Domande utili durante l’intervista}
\begin{itemize}
    \item A cosa stai pensando?
    \item Cosa pensi dovresti fare in questa schermata?
    \item Cosa ti aspetti che succeda cliccando questo bottone?
    \item Qual \`e il tuo obiettivo in questo momento?
    \item Ho notato che… a cosa stavi pensando in quel momento?
    \item Cosa stai cercando?
\end{itemize}

\end{document}

\end{document}
